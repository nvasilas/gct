\documentclass[a4paper,11pt]{article}
\usepackage[english,greek]{babel}
\usepackage[utf8]{inputenc}
\usepackage[T1]{fontenc}
\usepackage{textcomp}

\usepackage{amsmath, amsthm, amssymb}
\usepackage{bm}

\usepackage{libertine}

\usepackage{url}
\usepackage{hyperref}
\hypersetup{
    unicode=true,
    pdftitle={Geometrical Control Theory problem set 2},
    pdfauthor={Vasilas Nikos},
    pdfsubject={Geometrical Control Theory},
    pdfcreator={Vasilas Nikos},
    pdfproducer={Vasilas Nikos},
    pdfkeywords={control, nonlinear, math},
}

\usepackage{graphicx}
\usepackage[small,bf]{caption}
\usepackage{subcaption}

\usepackage{booktabs}

\usepackage
[a4paper, top=1.0in, bottom=1.25in, left=1.25in, right=1.25in]{geometry}
\headsep=20pt
\footskip=30pt

\usepackage{authoraftertitle}

\makeatletter
\renewcommand{\maketitle}{\bgroup\setlength{\parindent}{0pt}
\thispagestyle{plain}
\begin{flushleft}
    \textbf{\Huge{\@title}\\}
    \vspace{2em}
    \@author
    \vspace{2em}
\end{flushleft}\egroup
}
\makeatother

\usepackage{csquotes}
%\usepackage[backend=biber, bibencoding=auto, autolang=other]{biblatex}
%\addbibresource{bible.bib}

\newcommand{\tl}[1]{\textlatin{#1}}
\newcommand{\gr}[1]{\greektext{#1}}
\newcommand{\greek}{\selectlanguage{greek}}
\newcommand{\eng}{\selectlanguage{english}}

\usepackage{enumitem}
\newenvironment{alphenum}
{\begin{enumerate}
    [itemsep=-0.5ex, topsep=1.0ex, leftmargin=\parindent, align=left,
    labelwidth=!, label=\((\alph*)\)]}
{\end{enumerate}}

\newenvironment{romenum}
{\begin{enumerate}
    [itemsep=-5pt, leftmargin=*, labelsep=1em, align=left, labelwidth=!,label=\((\roman*)\)]}
{\end{enumerate}}

\usepackage{mathtools}
\newcommand\defeq{\coloneq}
\newcommand\eqdef{\eqcolon}

\newcommand\proofpt[1]{\({(#1)}\)}

\newcommand\inner[2]{\langle#1,\, #2\rangle}

\newcommand\dual[1]{#1^{*}}
\newcommand\algdual[1]{#1^{\#}}

\newcommand\lmaps{\mathcal{B}}

\newcommand\realf{R}
\newcommand\euclr[1]{\realf^{#1}}
\newcommand\deuclr[1]{\dual{(\euclr{#1})}}

\newcommand\restr[2]{#1|_#2}
\newcommand\compf{\mathbb{C}}
\newcommand\matrx[1]{\mathrm{#1}}
\newcommand\pcmap[1]{PC^{#1}}
\newcommand\cdiff[1]{C^{#1}}
\newcommand\metric{d}
\newcommand\di[2]{\metric(#1, #2)}

\newcommand{\vc}[1]{\bm{#1}} % this is a vector
\newcommand{\mt}[1]{\boldsymbol{#1}} % this is a matrix

\DeclareMathOperator\rank{rank}
\DeclareMathOperator\vspan{span}
\DeclareMathOperator\co{co}
\DeclareMathOperator\aff{aff}
\DeclareMathOperator\core{cor}
\DeclareMathOperator\icr{icr}
\DeclareMathOperator\diag{diag}
\DeclareMathOperator\minimize{minimize}
\DeclareMathOperator\argmin{arg\,min}
\DeclareMathOperator\der{D}
\DeclareMathOperator\qi{qi}
\DeclareMathOperator\qci{qci}
\DeclareMathOperator{\tr}{tr}

\newcommand\cseg[2]{[#1, #2]}
\newcommand\sseg[2]{[#1, #2)}
\newcommand\oseg[2]{(#1, #2)}

\newcommand\eps{\varepsilon}
\newcommand\dsum{\oplus}

\newcommand\norm[2][{}]{\|#2\|_{#1}}

\makeatletter
\newcommand\suchthat{%
    \@ifstar
    {\mathrel{}\middle|\mathrel{}}
    {\mid}%
}
\makeatother
\makeatletter
\newcommand\setbuild[2]{
    \@ifstar
    {\left\{#1\suchthat* #2\right\}}%
    {\{#1\suchthat #2\}}%
}
\makeatother
\newcommand\ball[2]{B_{#1}(#2)}
\newcommand\cball[2]{\closure{B}_{#1}(#2)}
\newcommand\oball[1]{B_{#1}}
\newcommand\ocball[1]{\bar{B}_{#1}}
\newcommand\compl[1]{#1^{\mathrm{c}}}
\newcommand\closure[2][.0]
{
    \mkern#1mu
    \overline{\mkern-#1mu#2}
}
\newcommand\interior[1]{{#1}^{\circ}}
\newcommand\boundary[1]{\partial#1}
\newcommand\voidset{\varnothing}

\newcommand\nulls[1]{\mathcal{N}(#1)}
\newcommand\range[1]{\mathcal{R}(#1)}

\newcommand\refthrm[2][]{Theorem \ifx\\#1\\\ref{#2}\else\ref{#2} \ref{#1}\fi}
\newcommand\refprop[2][]{Proposition \ifx\\#1\\\ref{#2}\else\ref{#2} \ref{#1}\fi}
\newcommand\reflemma[2][]{Lemma \ifx\\#1\\\ref{#2}\else\ref{#2} \ref{#1}\fi}

\renewcommand\theequation{\arabic{equation}}
\newtheoremstyle{thrm_style}
{} %spaceabove
{} %spacebelow
{\itshape} %bodyfont
{} %indent
{\bfseries} %headfont
{} %headpunctuation
{1em} %headspace
{\thmnumber{#2\kern1em}\thmname{#1}\thmnote{ (#3)}} %headspec
\theoremstyle{thrm_style}
\newtheorem{thrm}{\gr{Θεώρημα}}
\newtheorem{prop}[thrm]{\gr{Πρόταση}}
\newtheorem{cor}[thrm]{\gr{Πόρισμα}}
\newtheorem{lemma}[thrm]{\gr{Λήμμα}}

\newcommand\theoremname{}
\newtheorem{genericthrm}[thrm]{\theoremname}
\newenvironment{namedthrm}[1]
{\renewcommand\theoremname{#1}
\begin{genericthrm}}
{\end{genericthrm}}

\newtheoremstyle{defn_style}
{} %spaceabove
{} %spacebelow
{} %bodyfont
{} %indent
{\bfseries} %headfont
{} %headpunctuation
{1em} %headspace
{\thmnumber{#2\kern1em}\thmname{#1}\thmnote{ (#3)}} %headspec
\theoremstyle{defn_style}
\newtheorem{defn}[thrm]{\gr{Ορισμός}}
\newtheorem{exmp}[thrm]{\gr{Παράδειγμα}}

\newcommand\defnname{}
\newtheorem{genericdefn}[thrm]{\defnname}
\newenvironment{nameddefn}[1]
{\renewcommand\defnname{#1}
\begin{genericdefn}}
{\end{genericdefn}}

\newtheoremstyle{rem_style}
{} %spaceabove
{} %spacebelow
{} %bodyfont
{} %indent
{\bfseries} %headfont
{.} %headpunctuation
{.5em} %headspace
{\thmnumber{#2\kern1em}\thmname{#1}\thmnote{ (#3)}} %headspec
\theoremstyle{rem_style}
\newtheorem*{rem}{Remark}
\newtheorem*{note}{Note}

\renewcommand\qed{\unskip\nobreak\quad\qedsymbol}
\renewcommand\qedsymbol{\rule{1ex}{1.6ex}}

\newenvironment{exercise}[2][Άσκηση]
{\begin{trivlist}
    \item[\hskip \labelsep {\bfseries #1}\hskip \labelsep {\bfseries #2.}]}
{\end{trivlist}}

\newenvironment{solution}[1][Λύση]
{\begin{trivlist}
    \item[\hskip \labelsep {\bfseries #1}]}
{\end{trivlist}}


\begin{document}

\title{Γεωμετρική Θεωρία Ελέγχου}
\author{1\textsuperscript{η} Σειρά Ασκήσεων\\Βασίλας Νικόλαος}
\maketitle

\begin{exercise}{2014/15 1}
    Δώστε τη δομή πολλαπλότητας της σφαίρας
    \(S^n = \{ x_1^2 + x_2^2 + \dots + x^2_{n+1} = 1 \} \subset \mathbb{R}^{n+1}\)
    με τους δύο χάρτες που δίνονται από τη στερεογραφική προβολή από το Βόρειο
    και Νότιο πόλο στο επίπεδο \( z = 0 \). Δώστε τους τύπους για τις προβολές
    και τις αντίστροφες απεικονίσεις τους, καθώς και τη συνάρτηση αλλαγής χάρτη
    από το \( \mathbb{R}^n - \{0\} \) στο \( \mathbb{R}^n - \{0\} \) και δείξτε
    ότι είναι λεία.
\end{exercise}
\begin{solution}{2014/15 1}
    Μπορούμε να καλύψουμε τη σφαίρα \( S^n \) με τα ανοιχτά σύνολα
    \begin{align*}
        U_N &= \{ (x_1, x_2, \dots, x_{n+1}) \in S^{n}: x_{n+1} \neq 1 \} \\
        U_S &= \{ (x_1, x_2, \dots, x_{n+1}) \in S^{n}: x_{n+1} \neq -1 \}.
    \end{align*}
    Ορίζουμε τους χάρτες \( \phi_N: U_N \to \mathbb{R}^n \) ως τη στερεογραφική
    προβολή από το Βόρειο πόλο \( N = (0,\dots,0, 1) \) και
    \( \phi_S: U_S \to \mathbb{R}^n \) ως τη στερεογραφική προβολή από το Νότιο
    πόλο \( S = (0,\dots,0, -1) \).

    Η ευθεία που ορίζεται από το Βόρειο πόλο είναι
    \begin{equation*}
        x_{n+1} = a x_i + b, \quad i = 1, \dots, n.
    \end{equation*}
    Για το σημείο \( N = (0,\dots,0,1) \) βρίσκουμε \( b = 1 \) και όταν \(
    x_{n+1} = 0 \) για την \( i \) συντεταγμένη ισχύει
    \begin{equation*}
        0 = a \phi_N(x_i) + 1,
    \end{equation*}
    δηλαδή
    \begin{equation*}
        a = -\dfrac{1}{\phi_N(x_i)},
    \end{equation*}
    και τελικά προκύπτει
    \begin{equation*}
        \phi_N(x_i) = \dfrac{x_i}{1 - x_{n+1}}.
    \end{equation*}
    Επομένως για το χάρτη από το Βόρειο πόλο ισχύει
    \begin{align*}
        \phi_N: U_N &\to \mathbb{R}^n \\
        (x_1, \dots, x_{n+1}) &\mapsto
        \left( \dfrac{x_1}{1 - x_{n+1}}, \dots, \dfrac{x_n}{1 - x_{n+1}}
        \right).
    \end{align*}
    Ακολουθώντας την ίδια διαδικασία για το Νότιο πόλο εύκολα φαίνεται ότι
    ισχύει
    \begin{align*}
        \phi_S: U_S &\to \mathbb{R}^n \\
        (x_1, \dots, x_{n+1}) &\mapsto
        \left( \dfrac{x_1}{1 + x_{n+1}}, \dots, \dfrac{x_n}{1 + x_{n+1}}
        \right).
    \end{align*}
    Και οι δύο χάρτες \( \phi_N, \phi_S \) είναι ένα προς ένα και λείες συναρτήσεις καθώς
    έχει αφαιρεθεί το \( x_{n+1} = 1 \) από την πρώτη και το \( x_{n+1} = -1\)
    από τη δεύτερη. Η αντίστροφη απεικόνιση για το Βόρειο πόλο είναι
    \begin{align*}
        \phi_N^{-1}: \mathbb{R}^n &\to U_N \\
        (y_1, \dots, y_n) &\mapsto (x_1, \dots, x_{n+1}).
    \end{align*}
    Για κάθε συνιστώσα \( i = 1, \dots, n \) ισχύει
    \begin{equation}\label{eq:1415_ex_1_y}
        y_i = \dfrac{x_i}{1- x_{n+1}},
    \end{equation}
    ή
    \begin{equation*}
        x_i = y_i(1- x_{n+1}).
    \end{equation*}
    Επίσης, για τη μοναδιαία σφαίρα \( S^n \) ισχύει
    \begin{equation*}
        x_1^2 + \dots + x_n^2 + x_{n+1}^2 = 1,
    \end{equation*}
    όπου με αντικατάσταση της παραπάνω προκύπτει
    \begin{equation*}
        y_1^2(1 - x_{n+1})^2 + \dots + y_n^2(1 - x_{n+1})^2 = 1 - x^2_{n+1},
    \end{equation*}
    όπου διαιρώντας με τον όρο \( 1 - x_{n+1} \) προκύπτει
    \begin{equation*}
        y_1^2(1 - x_{n+1}) + \dots + y_n^2(1 - x_{n+1}) = 1 + x_{n+1},
    \end{equation*}
    και λύνοντας ως προς \( x_{n+1} \) έχουμε
    \begin{equation}\label{eq:1415_ex_1_x_n}
        x_{n+1} = \dfrac{y_1^2 + \dots + y_n^2 - 1}{y_1^2 + \dots + y_n^2 + 1}.
    \end{equation}
    Από τη σχέση \eqref{eq:1415_ex_1_y}, λύνοντας ως προς \( x_{n+1} \) έχουμε
    τη σχέση
    \begin{equation*}
        x_{n+1} = - \dfrac{x_i}{y_i} + 1.
    \end{equation*}
    Με αντικατάσταση της παραπάνω στη \eqref{eq:1415_ex_1_x_n} προκύπτει
    \begin{equation*}
        - \dfrac{x_i}{y_i} + 1 = \dfrac{y_1^2 + \dots + y_n^2 - 1}{y_1^2 + \dots
        + y_n^2 + 1},
    \end{equation*}
    ή τελικά
    \begin{equation}\label{eq:1415_ex_1_x_i}
        x_i = \dfrac{2y_i}{y_1^2 + \dots + y_n^2 + 1},
    \end{equation}
    για \( i = 1, \dots, n \). Επομένως για την αντίστροφη απεικόνιση για το
    Βόρειο πόλο από τις σχέσεις \eqref{eq:1415_ex_1_x_i} και
    \eqref{eq:1415_ex_1_x_n} ισχύει
    \begin{align*}
        \phi_N^{-1}(y_1, \dots, y_n)&=(x_1, \dots, x_{n+1})\\
        &= \left( \dfrac{2y_1}{\sum y_i^2 + 1}, \dots
            ,\dfrac{2y_n}{\sum y_i^2 + 1},
        \dfrac{\sum y_i^2 - 1}{\sum y_i^2 + 1} \right).
    \end{align*}
    Με ακριβώς την ίδια λογική κάνοντας τις πράξεις καταλήγουμε για την
    αντίστροφη απεικόνιση για το Νότιο πόλο
    \begin{align*}
        \phi_S^{-1}(y_1, \dots, y_n)&=(x_1, \dots, x_{n+1})\\
        &= \left( \dfrac{2y_1}{\sum y_i^2 + 1}, \dots
            ,\dfrac{2y_n}{\sum y_i^2 + 1},
        \dfrac{1 - \sum y_i^2}{\sum y_i^2 + 1} \right).
    \end{align*}
    Η συνάρτηση αλλαγής χάρτη είναι
    \begin{equation*}
        (\phi_S \circ \phi_N^{-1}) (y_1, \dots, y_n)=
        \phi_S \left( \dfrac{2y_1}{\sum y_i^2 + 1}, \dots
            ,\dfrac{2y_n}{\sum y_i^2 + 1},
        \dfrac{\sum y_i^2 - 1}{\sum y_i^2 + 1} \right).
    \end{equation*}
    Για τη συνιστώσα \( i = 1,\dots,n \) ισχύει
    \begin{align*}
        \phi_S \left( \dfrac{2y_i}{\sum y_i^2 + 1} \right) &=
        \dfrac{\dfrac{2y_i}{\sum y_i^2 + 1}}
        {1 + \dfrac{\sum y_i^2 - 1}{\sum y_i^2 + 1}}
        = \dfrac{\dfrac{2y_i}{\sum y_i^2 + 1}}
        {\dfrac{2\sum y_i^2}{\sum y_i^2 + 1}} \\
        &= \dfrac{y_i}{\sum y_i^2},
    \end{align*}
    που είναι λεία στο \( \mathbb{R}^n - \{0\} \) και δίνει την απεικόνιση
    \begin{equation*}
        \phi_S \circ \phi_N^{-1}:\phi_N(U_N \cap U_S) = \mathbb{R}^n - \{0\} \to
        \phi_S(U_N \cap U_S) = \mathbb{R}^n - \{0\}.
    \end{equation*}
    Αντίστοιχα
    \begin{equation*}
        (\phi_N \circ \phi_S^{-1}) (y_1, \dots, y_n)=
        \left( \dfrac{y_1}{\sum y_i^2}, \dots,
        \dfrac{y_n}{\sum y_i^2}\right),
    \end{equation*}
    που επίσης είναι λεία στο \( \mathbb{R}^n - \{0\} \) και δίνει την απεικόνιση
    \begin{equation*}
        \phi_N \circ \phi_S^{-1}:\phi_S(U_N \cap U_S) = \mathbb{R}^n - \{0\} \to
        \phi_N(U_N \cap U_S) = \mathbb{R}^n - \{0\}.
    \end{equation*}
\end{solution}

\begin{exercise}{2015/15 2}
    Διανυσματικά πεδία σε \emph{πολλαπλότητες}:
    \begin{enumerate}
        \item \textbf{Σφαίρα}: Θεωρούμε δύο τρόπους για να ορίσουμε ΔΠ στη
            σφαίρα \( S^2 \) (και οι δύο εκμεταλλεύονται το γεγονός ότι έχουμε
            εμβύθιση στον Ευκλείδειο χώρο \( \mathbb{R}^3 \) ).
            Ο πρώτος είναι με χρήση πολικών συντεταγμένων
            \( (\theta, \phi) \mapsto \vc{r}(\theta, \phi) \)
            (εδώ \( \theta \in (0, \pi) \) και \( \phi \in (0, 2\pi) \)).
            Δείξτε ότι τα ΔΠ \( \vc{r}_{\theta}, \vc{r}_{\phi} \) δίνουν
            βάση του εφαπτόμενου χώρου στη σφαίρα σε κάθε σημείο όπου οι
            πολικές συντεταγμένες ορίζονται. Τα ΔΠ αυτά είναι ίδια με τα ΔΠ
            \( \frac{\partial}{\partial \theta} \) και \( \frac{\partial}{\partial \phi} \)
            που μελετήσαμε στη θεωρία ή όχι; Ποιες είναι οι λύσεις των ΔΠ αυτών;

            Ένα ΔΠ στη σφαίρα δίνεται λοιπόν ως \( X = a(\theta,
            \phi)\vc{r}_{\theta} + b(\theta, \phi)\vc{r}_{\phi} \). Πως πρέπει
            να περιορίσουμε τις συναρτήσεις \( a, b \) έτσι ώστε να είναι καλά
            ορισμένο, λείο ΔΠ στη σφαίρα;

            Ο δεύτερος τρόπος είναι να παρατηρήσουμε ότι το ΔΠ \( \vc{F}_h =
            (-y, x, 0) \) στο \( \mathbb{R}^3 \) αφήνει αναλλοίωτη τη σφαίρα \(
            S^2 \) και κάθε σφαίρα με κέντρο το \( 0 \), καθώς \( \vc{r} \cdot
            \vc{F} = 0 \). Δείξτε ότι οι λύσεις του πεδίου αυτού συμπίπτουν με
            τις λύσεις του \( \vc{r}_{\phi} \). Εάν ορίσουμε ένα δεύτερο ΔΠ από
            το εξωτερικό γινόμενο, \( \vc{F}_u = \vc{F}_h \times \vc{r} \), τότε και
            το \( \vc{F}_u \) εφάπτεται της σφαίρας. Εξηγήστε προσεκτικά πως θα
            μπορούσαμε να ορίσουμε ένα γενικό ΔΠ σε κάποιο υποσύνολο της σφαίρας
            από τα \( \vc{F}_h, \vc{F}_u \). Συγκρίνετε τις δύο μεθόδους.
        \item \textbf{Τόρος}: Εδώ θεωρούμε τον τόρο ως χώρο πηλίκου \(
            \mathbb{R}^2/\mathbb{Z}^2 \). Εξηγήστε γιατί κάθε ΔΠ στο επίπεδο το
            οποίο είναι \(2\pi\)-περιοδικό στο \(x\) και στο \(y\) δίνει λείο ΔΠ
            στον τόρο.

            Με χρήση λογισμικού, περιγράψτε τη δυναμική του ΔΠ
            \begin{equation*}
                \vc{F} = \cos{(x)}\sin{(2x)}\vc{i} + \sin{(x + 2y)}\vc{j}.
            \end{equation*}

            Τέλος, μπορείτε να βρείτε αναλυτική μορφή για ΔΠ στον τόρο που
            αντιστοιχεί στην κλίση του πεδίου που δίνεται από τη συνάρτηση
            ύψους, όταν ο τόρος θεωρηθεί ως τοποθετημένος κατακόρυφα στο \(
            \mathbb{R}^3 \);
    \end{enumerate}
\end{exercise}
\begin{solution}{2014/15 2}
    Με τη χρήση πολικών συντεταγμένων έχουμε
    \begin{equation*}
        (\theta, \phi) \mapsto \vc{r}(\theta, \phi) =
        \begin{pmatrix}
            x(\theta, \phi)\\
            y(\theta, \phi)\\
            z(\theta, \phi)
        \end{pmatrix}=
        \begin{pmatrix}
            \sin{\theta}\cos{\phi}\\
            \sin{\theta}\sin{\phi}\\
            \cos{\theta}
        \end{pmatrix}.
    \end{equation*}
    Τα διανυσματικά πεδία \( \vc{r}_{\theta}, \vc{r}_{\phi} \) είναι
    \begin{align*}
        \vc{r}_{\theta} &= \frac{\partial}{\partial \theta} =
        \frac{\partial x}{\partial \theta}\frac{\partial}{\partial x} +
        \frac{\partial y}{\partial \theta}\frac{\partial}{\partial y} +
        \frac{\partial z}{\partial \theta}\frac{\partial}{\partial z} \\
        &=
        \cos{\theta}\cos{\phi}\frac{\partial}{\partial x} +
        \cos{\theta}\sin{\phi}\frac{\partial}{\partial y} +
        -\sin{\theta}\frac{\partial}{\partial z},
    \end{align*}
    και
    \begin{align*}
        \vc{r}_{\phi} &= \frac{\partial}{\partial \phi} =
        \frac{\partial x}{\partial \phi}\frac{\partial}{\partial x} +
        \frac{\partial y}{\partial \phi}\frac{\partial}{\partial y} +
        \frac{\partial z}{\partial \phi}\frac{\partial}{\partial z} \\
        &=
        -\sin{\theta}\sin{\phi}\frac{\partial}{\partial x} +
        \sin{\theta}\cos{\phi}\frac{\partial}{\partial y}.
    \end{align*}
    Τα διανυσματικά πεδία \( \vc{r}_{\theta}, \vc{r}_{\phi} \) δίνουν βάση
    του εφαπτόμενου χώρου στη σφαίρα σε κάθε σημείο όπου ορίζονται οι πολικές
    συντεταγμένες γιατί είναι γραμμικώς ανεξάρτητα
    \begin{equation*}
        \vc{r}_{\theta} \cdot \vc{r}_{\phi} =
        \left(\cos{\theta}\cos{\phi}\right) \left(-\sin{\theta}\sin{\phi}\right)+
        \left(\cos{\theta}\sin{\phi}\right) \left(\sin{\theta}\cos{\phi}\right)
        = 0,
    \end{equation*}
    και κάθε διάνυσμα είναι γραμμικός συνδυασμός των παραπάνω διανυσμάτων. Οι
    λύσεις των διανυσματικών πεδίων είναι οι ολοκληρωτικές καμπύλες.

    Ένα διανυσματικό πεδίο είναι λείο αν ισχύει το παρακάτω.
    \begin{lemma}[Λείο διανυσματικό πεδίο σε χάρτη, λήμμα 14.1
        του \tl{Tu} \cite{tu2010introduction}]
        Έστω χάρτης \( (U, \phi) = (U, x_1, \dots, x_n) \) στην
        πολλαπλότητα \( M \). Ένα διανυσματικό πεδίο \( X = \sum a_i
        \partial/\partial x_i \) στο \(U\) είναι λείο αν και μόνο αν οι
        συναρτήσεις \( a_i \) είναι λείες στο \( U \).
    \end{lemma}
    Επομένως, το
    \begin{equation*}
        X = a(\theta, \phi)\vc{r}_{\theta} + b(\theta, \phi)\vc{r}_{\phi},
    \end{equation*}
    είναι λείο όταν οι συναρτήσεις \(a, b\) είναι λείες.

    Το διανυσματικό πεδίο \( \vc{F}_h \) είναι
    \begin{equation*}
        \vc{F}_h = -y \frac{\partial}{\partial x} + x \frac{\partial}{\partial
        y},
    \end{equation*}
    και οι λύσεις του είναι
    \begin{equation*}
        \dot{x}(t) = -y, \quad \dot{y}(t) = x.
    \end{equation*}
    Όμως από τις πολικές συντεταγμένες ισχύει \( x = \sin{\theta}\cos{\phi},
    y = \sin{\theta}\sin{\phi} \) και άρα οι παραπάνω σχέσεις γίνονται
    \begin{equation*}
        \dot{x}(t) = -\sin{\theta}\sin{\phi} , \quad \dot{y}(t) = \sin{\theta}\cos{\phi},
    \end{equation*}
    που συμπίπτουν τις λύσεις του \( \vc{r}_{\phi} \).
    Το \( \vc{F}_h \) μας δίνει διανυσματικό πεδίο στο επίπεδο που \( z = 0 \). Τα
    \( \vc{F}_u \) είναι κάθετο σε αυτό, καθώς \( \vc{r} \cdot \vc{F}_h = 0 \)
    και \( \vc{F}_u = \vc{F}_h \times \vc{r} \). Επομένως, παίρνοντας αυτά τα δύο ως
    βάση μπορούμε να ορίσουμε ΔΠ σε κάθε υποσύνολο της σφαίρας που ορίζονται
    \( \vc{F}_h \) και \( \vc{F}_u \).

    Ο δύο διαστάσεων τόρος \(T^2 = S^1 \times S^1\) μπορεί να παρουσιαστεί στο
    τετράγωνο ως \( \{x,y: 0 \leq x \leq 2\pi, 0 \leq y \leq 2\pi \} \),
    ενώνοντας τις αντίθετες πλευρές, δηλαδή τα σημεία \( (0, y) \) και \( (2\pi,
    y) \), καθώς και τα σημεία \( (x,0) \) και \( (x, 2\pi) \). Επίσης, μπορούμε
    να δούμε τον τόρο ως χώρο πηλίκου \( \mathbb{R}^2 / \mathbb{Z}^2 \) με τις
    ακέραιες συνιστώσες πολλαπλασιασμένες με \(2\pi\), δηλαδή \(T^2 =
        \mathbb{R}^2/ 2\pi\mathbb{Z}^2 = \{ (x, y) \in \mathbb{R}^2 \mod 2\pi \}
    \). Επομένως, το επίπεδο \(\mathbb{R}^2\) καλύπτει τον τόρο τοπικά
    διαφορομορφικά. Μέσω της κάλυψης \(\mathbb{R}^2 \to T^2\) μπορούμε να
    μεταφέρουμε οποιαδήποτε εικόνα του τόρου στο επίπεδο. Οι λείες συναρτήσεις
    στον τόρο αντιστοιχούν σε \(2\pi-\)περιοδικές λείες συναρτήσεις στο επίπεδο.
    Έτσι μπορούμε να πούμε ότι κάθε διανυσματικό πεδίο στον τόρο δίνει πεδίο στο
    επίπεδο περιοδικό με περίοδο \(2\pi\) και στις δύο συντεταγμένες.
    Αντιστρόφως, κάθε διανυσματικό πεδίο \(2\pi\) περιοδικό και στις δύο
    συντεταγμένες στο επίπεδο, δίνει λείο διανυσματικό πεδίο στον τόρο.
\end{solution}

\begin{exercise}{2014/15 3}
    Εάν \( M^{n \times n} \) είναι ο διανυσματικός χώρος όλων των \( n \times
    n\) πινάκων με πραγματικά στοιχεία και \( \det : M \to \mathbb{R}
    \) είναι η απεικόνιση ορίζουσας, ποιο από τα δύο υποσύνολα του
    \( M, N = \{ \det A = 0\} \) και \( \text{\tl{GL}}(n) = \{
    \det A \neq 0 \} \) είναι πολλαπλότητα; (προσοχή στη διάσταση
    \( n \)).
    Δείξτε ότι το υποσύνολο \( \text{\tl{SL}}(n) = \{ \det A = 1 \} \)
    δίνει υπο-πολλαπλότητα (υπόδειξη: δείξτε ότι είναι κανονική τιμή της
    συνάρτησης \( \det \) ).
\end{exercise}
\begin{solution}{2014/15 3}
    Αρχικά ο διανυσματικός χώρος \( M^{n\times n} \) με πραγματικά στοιχεία είναι
    ισομορφικός με το \( \mathbb{R}^{n^2} \). Η γενική γραμμική ομάδα είναι
    \begin{equation*}
        \text{\tl{GL}}(n) = \{A \in M^{n \times n}: \det A \neq 0 \} =
        \det{}^{-1}(\mathbb{R} - \{0\}).
    \end{equation*}
    Όμως η ορίζουσα έχει πολυωνυμική μορφή και είναι επομένως συνεχής συνάρτηση.
    Επίσης, το σύνολο \( \mathbb{R} - \{0\} \) είναι ανοιχτό σύνολο. Άρα αφού η
    αντίστροφη εικόνα συνεχής συνάρτησης είναι ανοιχτό σύνολο, από τον ορισμό της συνέχειας
    προκύπτει ότι \( \text{\tl{GL}}(n) \) είναι ανοιχτό υποσύνολο του \(
    \mathbb{R}^{n \times n} \simeq \mathbb{R}^{n^2}\). Φυσικά δίνοντας τη
    συνήθη τοπολογία στο \( \mathbb{R}^{n^2} \) είναι πολλαπλότητα διάστασης
    \( n^2 \), και συνεπώς η \( \text{\tl{GL}}(n) \) ως ανοιχτό υποσύνολο
    πολλαπλότητας είναι πολλαπλότητα ίδιας διάστασης.

    Η ειδική γραμμική ομάδα είναι
    \begin{equation*}
        \text{\tl{SL}}(n) = \{A \in M^{n \times n}: \det A = 1 \} =
        \det{}^{-1}(\{1\}).
    \end{equation*}
    Άρα αρκεί να δείξουμε ότι το \( 1 \) είναι κανονική τιμή της συνάρτησης
    \( \det \). Η ορίζουσα ενός τετράγωνου \( A \in M^{n \times n}(\mathbb{R}) \)
    πίνακα μπορεί να εκφραστεί ως ανάπτυγμα με ελάσσονες ορίζουσες
    (\tl{cofactor expansion}) ως προς τη γραμμή \( i \) με \((1 \leq i \leq n) \)
    \begin{equation}\label{eq:1415_ex_3_det}
        \det A = \sum_{j = 1}^n
        (-1)^{i+j}A_{ij}\det\bar{A}_{ij}=
        \sum_{j = 1}^n (-1)^{i+j}A_{ij}M_{ij},
    \end{equation}
    όπου \( \bar{A}_{ij} \) δηλώνει τον \( (n - 1)\times(n - 1) \) πίνακα που
    προκύπτει αφαιρώντας την \( i-\)γραμμή και την \(j-\)στήλη
    από τον \(A\), \(A_{ij}\) είναι το \(i,j\) στοιχείο του πίνακα \(A\) και \(
    M_{ij} = \det \bar{A}_{ij}\) είναι η ελάσσων ορίζουσα.

    Επομένως παίρνοντας την παράγωγο έχουμε
    \begin{equation*}
        \dfrac{\partial \det A}{\partial A_{ij}} =
        (-1)^{i+j}M_{ij},
    \end{equation*}
    για να είναι ο \( A \) κρίσιμη τιμή της συνάρτησης \( \det \) θα πρέπει κάθε
    ελάσσονα ορίζουσα \( M_{ij} \) να είναι μηδέν. Όμως αυτό σημαίνει
    ότι η ορίζουσα του \( A \) θα είναι μηδέν, όπως φαίνεται από τη σχέση
    \eqref{eq:1415_ex_3_det}, άτοπο. Έτσι κάθε πίνακας στο \(\text{\tl{SL}}(n)\) είναι
    κανονική τιμή της \( \det \). Τελικά, έχουμε ότι \( \det \) είναι λεία
    απεικόνιση των πολλαπλοτήτων \( \mathbb{R}^{n^2} \) και \( \mathbb{R} \)
    με διαστάσεις \( n^2 \) και \( 1 \) αντίστοιχα και από το θεώρημα κανονικής τιμής
    προκύπτει ότι \(\text{\tl{SL}}(n)\) είναι υπο-πολλαπλότητα διάστασης \( n^2 - 1 \).

    Το σύνολο \( N = \{A \in M^{n \times n}: \det A = 0 \} \) δεν είναι
    πολλαπλότητα. Ακόμα ισούται \( N = \det{}^{-1}(\{0\}) \), με το \(0\) να
    είναι κρίσιμη τιμή της ορίζουσας, το οποίο εύκολα φαίνεται ακολουθώντας την
    παραπάνω διαδικασία. Μπορούμε να πούμε ότι το σύνολο των σημείων που ανήκουν
    στο \(N\) δημιουργούν ένα σημείο ανωμαλίας (\tl{singularity}) στο χώρο
    \(\mathbb{R}^{n^2} \). Διαισθητικά καταλαβαίνουμε ότι αυτό δε μπορεί να είναι
    λεία πολλαπλότητα. Ένα παράδειγμα που είναι ίδιας φύσεως, είναι η ένωση του
    \( x-\)άξονα και του \(y-\)άξονα στο \( \mathbb{R}^2 \). Αφαιρώντας το \((0, 0)\)
    από τη γειτονία θα μας δώσει \(4\) συνεκτικές συνιστώσες κάθε μία από αυτές
    ομοιομορφική με το ανοιχτό διάστημα, ενώ το \( \mathbb{R} - \{0\} \) μας δίνει το
    πολύ δύο συνιστώσες, οπότε δεν μπορούμε να έχουμε ομοιομορφισμό από το
    \( \mathbb{R}^2 -\{(0,0)\}\) στο \( \mathbb{R} -\{0\}\) και άρα δεν είναι πολλαπλότητα.
\end{solution}

\begin{exercise}{2014/15 4}
    Δώστε τους τρεις πρώτους όρους της σειράς \tl{Taylor} για τις συνθέσεις
    γραμμικών ροών:
    \begin{equation*}
        e^{tA}e^{tB} \quad \text{\gr{και}} \quad e^{-tA}e^{-tB}e^{tA}e^{tB}.
    \end{equation*}
    Στην πρώτη περίπτωση, εκφράστε την απάντησή σας μέσω των όρων της σειράς
    \tl{Taylor} της \( e^{t(A+B)} \) και κατάλληλων αγκυλών \tl{Lie} (στην
    άλγεβρα \tl{Lie} πινάκων). Τι εικάζετε ότι θα έχουμε συνολικά; Δείξτε ότι
    εάν \( AB = BA \), τότε \( e^{tA}e^{tB} = e^{t(A+B)}, \forall t \).
\end{exercise}
\begin{solution}{2014/15 4}
    Το ανάπτυγμα \tl{Taylor} για τους τρεις πρώτους όρους των γραμμικών ροών
    είναι
    \begin{align*}
        e^{tA}e^{tB} &=
        \left( I + tA + \dfrac{t^2}{2}A^2 + \dots \right)
        \left( I + tB + \dfrac{t^2}{2}B^2 + \dots \right)\\
        &= I + tB + \dfrac{t^2}{2}B^2 + tA + t^2AB + \dfrac{t^3}{2}AB^2\\
        &\quad\qquad \qquad
        +\dfrac{t^2}{2}A^2 + \dfrac{t^3}{2}A^2B + \dfrac{t^4}{4}A^2B^2 + \dots\\
        &= I + t(A + B) + \dfrac{t^2}{2}(A^2 + B^2 + 2AB) + \dots,
    \end{align*}
    που αν θεωρήσουμε ότι η αγκύλη \tl{Lie} είναι
    \begin{equation*}
        [A, B] = BA - AB,
    \end{equation*}
    τότε η παραπάνω γράφεται
    \begin{align}\label{eq:1415_ex_4_etab_full}
        e^{tA}e^{tB} &=
        I + t(A + B) - \frac{t^2}{2}\left( -A^2 - B^2 - 2AB - BA + BA \right) +
        \dots \nonumber \\
        &=I + t(A + B) - \frac{t^2}{2}[A, B] + \frac{t^2}{2}(A + B)^2 + \dots,
    \end{align}
    ή κρατώντας τους όρους πρώτης τάξης
    \begin{align}\label{eq:1415_ex_4_etab}
        e^{tA}e^{tB} &=
        I + t(A + B) - \frac{t^2}{2}[A, B]\nonumber\\
        &= e^{t(A+B) - \frac{t^2}{2}[A,B]}.
    \end{align}
    Η σύνθεση της γραμμικής ροής
    \begin{equation*}
        e^{-tA}e^{-tB}e^{tA}e^{tB},
    \end{equation*}
    μπορεί να υπολογισθεί χρησιμοποιώντας τη σχέση \eqref{eq:1415_ex_4_etab}. Ο
    πρώτος όρος είναι
    \begin{equation*}
        e^{-tA}e^{-tB} = e^{-t(A + B) + \frac{t^2}{2}[A, B]},
    \end{equation*}
    και ο δεύτερος είναι
    \begin{equation*}
        e^{tA}e^{tB} = e^{t(A + B) + \frac{t^2}{2}[A, B]}.
    \end{equation*}
    Θέτοντας
    \begin{equation*}
        C = -(A + B) - \frac{t}{2}[A, B] \quad \text{\gr{και}} \quad
        D = (A + B) - \frac{t}{2}[A, B],
    \end{equation*}
    προκύπτει για τη σύνθεση της γραμμικής ροής
    \begin{align*}
        e^{-tA}e^{-tB}e^{tA}e^{tB} &=
        e^{-t(A + B) - \frac{t^2}{2}[A, B]}e^{t(A + B) - \frac{t^2}{2}[A, B]}\\
        &= e^{tC}e^{tD} = e^{t(C + D) - \frac{t^2}{2}[C, D]}.
    \end{align*}
    Για τον πρώτο όρο του εκθετικού πίνακα έχουμε
    \begin{align*}
        t(C + D) &=
        -t(A + B) - \frac{t^2}{2}[A, B] + t(A+ B) - \frac{t^2}{2}[A,B] \\
        &= -t^2[A,B].
    \end{align*}
    Η αγκύλη \tl{Lie} του δεύτερου όρου είναι
    \begin{align*}
        [C,D] &=
        \left((A+B) - \frac{t}{2}[A,B]\right)
        \left(-(A+B) - \frac{t}{2}[A,B]\right)\\
        &\quad-
        \left(-(A+B) - \frac{t}{2}[A,B]\right)
        \left((A+B) - \frac{t}{2}[A,B]\right)\\
        &=-t(A+B)[A,B] + t[A,B](A+B),
    \end{align*}
    και άρα ο δεύτερος όρος είναι
    \begin{equation*}
        -\frac{t^2}{2}[C,D] =
        \frac{t^3}{2}(A+B)[A,B] - \frac{t^3}{2}[A,B](A+B).
    \end{equation*}
    Τελικά, προκύπτει
    \begin{equation*}
        e^{-tA}e^{-tB}e^{tA}e^{tB} =
        e^{-t^2[A,B] +
        \frac{t^3}{2}(A+B)[A,B] - \frac{t^3}{2}[A,B](A+B)},
    \end{equation*}
    ή στην ισοδύναμη μορφή
    \begin{equation*}
        e^{-tA}e^{-tB}e^{tA}e^{tB} =
        e^{-t^2[A,B] + O(t^3)}.
    \end{equation*}
    Από τη σειρά \tl{Taylor} για τους τρεις πρώτους όρους τους αθροίσματος της
    γραμμικής ροής ισχύει
    \begin{equation*}
        e^{t(A + B)} = I + t(A + B) + \frac{t^2}{2}(A + B)^2 + \dots,
    \end{equation*}
    όπου συγκρίνοντας την παραπάνω με τη σχέση
    \eqref{eq:1415_ex_4_etab_full} και εάν
    \( AB=BA \) τότε \( [A,B] = 0 \) και φαίνεται ότι ισχύει
    \begin{equation*}
        e^{tA}e^{tB} = e^{t(A + B)}
    \end{equation*}
    για κάθε \( t \). Επίσης, αυτό μπορεί να δειχθεί από το διωνυμικό θεώρημα
    ,που ισχύει επειδή οι πίνακες αντιμετατίθενται,
    \begin{equation*}
        (A + B)^n = \sum_{k=0}^n \begin{pmatrix}n\\k\end{pmatrix}A^kB^{n-k},
    \end{equation*}
    όπου
    \begin{equation*}
        \begin{pmatrix}n\\k\end{pmatrix} = \dfrac{n!}{k!(n - k)!}.
    \end{equation*}
    Έτσι έχουμε
    \begin{align*}
        e^{t(A+B)} &= \sum_{n=0}^{\infty}\frac{t^n}{n!}(A+B)^n\\
        &= \sum_{n=0}^{\infty}\frac{t^n}{n!}\sum_{k=0}^n\begin{pmatrix}n\\k\end{pmatrix}A^kB^{n-k}\\
        &=\sum_{0 \leq k \leq n < \infty}\frac{t^n}{k!(n - k)!}A^kB^{n-k},
    \end{align*}
    και αν θέσουμε \( n - k = l\) τότε
    \begin{align*}
        e^{t(A+B)} &=\sum_{k=0}^{\infty}\sum_{l=0}^{\infty}
        \frac{t^k}{k!}\frac{t^l}{l!}A^k B^{l} \\
        &=\sum_{k=0}^{\infty}\frac{t^k}{k!}A^k
        \sum_{l=0}^{\infty}\frac{t^l}{k!}B^{l} \\
        &=e^{tA}e^{tB}.
    \end{align*}
    Συνεπώς, οι όροι μεγαλύτερης τάξης που βλέπουμε στη
    \eqref{eq:1415_ex_4_etab_full} σχετίζονται με κάποια αγκύλη \tl{Lie} και για
    αυτό μηδενίζονται όταν ισχύει η αντιμεταθετικότητα.
\end{solution}

\begin{exercise}{2015/16 1β}
    Δώστε δομή πολλαπλότητας στον προβολικό χώρο \( \mathbb{R}P^n \) (του
    συνόλου των ευθειών στο \( \mathbb{R}^{n+1} \). Πως μπορούμε να ορίσουμε
    διανυσματικό πεδίο στο χώρο αυτό; Δώστε ένα παράδειγμα Δ.Π. στον \(
    \mathbb{R}P^2 \).
\end{exercise}
\begin{solution}{2015/16 1β}
    Η ανάπτυξη της δομής πολλαπλότητας στον προβολικό χώρο \( \mathbb{R}P^n \) στηρίχθηκε στο
    βιβλίο του \tl{Tu} \cite{tu2010introduction} (κεφάλαιο \(7\)).

    Ορίζουμε μία σχέση ισοδυναμίας στο \( \mathbb{R}^{n+1} - \{0\} \) ως
    \begin{equation*}
        x \sim y \iff y = tx \text{\gr{ για κάποιο μη μηδενικό πραγματικό αριθμό
        }} t,
    \end{equation*}
    όπου \( x,y \in \mathbb{R}^{n+1} - \{0\} \). Ο πραγματικός προβολικός χώρος
    \( \mathbb{R}P^n = (\mathbb{R}^{n+1} - \{0\})/_{\sim}\) είναι ο χώρος πηλίκου
    του \( \mathbb{R}^{n+1} - \{0\} \)
    με την παραπάνω σχέση ισοδυναμίας. Συμβολίζουμε την ισοδύναμη κλάση του
    σημείου \( (a_0, \dots, a_n ) \in \mathbb{R}^{n+1} - \{0\} \) με \(
    [a_0,\dots,a_n] \) και \( \pi: \mathbb{R}^{n+1} - \{0\} \to \mathbb{R}P^n \)
    η προβολή. Ονομάζουμε \( [a_0, \dots, a_n] \) τις ομογενείς συντεταγμένες
    στο \( \mathbb{R}P^n \).

    Γεωμετρικά, δύο μη μηδενικά σημεία του \( \mathbb{R}^{n+1} \) είναι
    ισοδύναμα αν και μόνο αν βρίσκονται στην ίδια ευθεία που περνάει από την
    αρχή, έτσι ο \( \mathbb{R}P^n \) μπορεί να ερμηνευτεί ως το σύνολο όλων των
    ευθειών που περνάν από την αρχή του \( \mathbb{R}^{n+1} \). Κάθε ευθεία που
    περνά από την αρχή του \( \mathbb{R}^{n+1} \) τέμνει τη μοναδιαία σφαίρα \(
    S^n\) σε ένα ζεύγος αντιδιαμετρικών σημείων και αντιστρόφως, κάθε ζεύγος
    αντιδιαμετρικών σημείων στο \( S^n \) καθορίζει μία ξεχωριστή ευθεία που
    περνά από την αρχή.

    Έστω \( [a_0, \dots, a_n] \) είναι οι ομογενείς συντεταγμένες του
    προβολικού χώρου \( \mathbb{R}P^n \). Η συνθήκη \( a_0 \neq 0 \) είναι
    ανεξάρτητη της επιλογής του αντιπροσώπου για \( [a_0, \dots, a_n] \) και
    επομένως έχει νόημα στο \( \mathbb{R}P^n \) και ορίζουμε
    \begin{equation*}
        U_0 = \{ [a_0, \dots, a_n] \in \mathbb{R}P^n: a_0 \neq 0 \}.
    \end{equation*}
    Αντίστοιχα, για κάθε \( i = 1, \dots, n \) θέσε
    \begin{equation*}
        U_i = \{ [a_0, \dots, a_n] \in \mathbb{R}P^n: a_i \neq 0 \}.
    \end{equation*}
    Όρισε
    \begin{equation*}
        \phi_0: U_0 \to \mathbb{R}^n
    \end{equation*}
    από
    \begin{equation*}
        [a_0, \dots, a_n] \mapsto \left( \frac{a_1}{a_0}, \dots,
        \frac{a_n}{a_0} \right).
    \end{equation*}
    Η απεικόνιση έχει συνεχή αντίστροφη
    \begin{equation*}
        (b_1, \dots, b_n) \mapsto [1, b_1, \dots, b_n]
    \end{equation*}
    και είναι επομένως ομοιομορφισμός. Αντίστοιχα, υπάρχουν ομοιομορφισμοί για
    κάθε \( i = 1, \dots, n \)
    \begin{align*}
        \phi_i: U_i &\to \mathbb{R}^n,\\
        [a_0, \dots, a_n] &\mapsto \left( \frac{a_1}{a_i}, \dots,
        \widehat{\frac{a_i}{a_i}}, \dots,\frac{a_n}{a_i} \right),
    \end{align*}
    όπου το σύμβολο \( \, \widehat{\,} \, \) πάνω από το \( a_i/a_i \) σημαίνει ότι το
    στοιχείο θα παραλειφθεί. Αυτό αποδεικνύει ότι ο \( \mathbb{R}P^n \) είναι
    τοπικά Ευκλείδειος με \( (U_i, \phi_i) \) ως χάρτες.

    Στην τομή \( U_0 \cap U_1 \), έχουμε \( a_0 \neq 0 \) και \( a_1 \neq 0 \),
    και δύο τοπικά συστήματα συντεταγμένων
    \begin{align*}
        [a_0, a_1, a_2, \dots, a_n] &\mapsto
        \left( \frac{a_1}{a_0}, \dots,
        \frac{a_2}{a_0}, \dots,\frac{a_n}{a_0} \right) \\
        [a_0, a_1, a_2, \dots, a_n] &\mapsto
        \left( \frac{a_0}{a_1}, \dots,
        \frac{a_2}{a_1}, \dots,\frac{a_n}{a_1} \right).
    \end{align*}
    Αναφέρουμε τις αλλαγές του συστήματος συντεταγμένων στο \( U_0 \) ως \(
    x_1, \dots, x_n \), και του συστήματος συντεταγμένων στο \( U_1 \) ως \(
    y_1, \dots, y_n \). Στο \( U_0 \) ισχύει
    \begin{equation*}
        x_i = \frac{a_i}{a_0}, \quad i = 1, \dots, n,
    \end{equation*}
    και στο \( U_1 \),
    \begin{equation*}
        y_1 = \frac{a_0}{a_1}, \quad y_2 = \frac{a_2}{a_1}, \quad \dots,
        \quad y_n = \frac{a_n}{a_1}.
    \end{equation*}
    Έτσι στο \( U_0 \cap U_1 \),
    \begin{equation*}
        y_1 = \frac{1}{x_1}, \quad
        y_2 = \frac{x_2}{x_1}, \quad
        y_3 = \frac{x_3}{x_1}, \quad
        \dots, \quad
        y_n = \frac{x_n}{x_1},
    \end{equation*}
    επομένως
    \begin{equation*}
        (\phi_1 \circ \phi_0^{-1})(x) =
        \left(
            \frac{1}{x_1},
            \frac{x_2}{x_1},
            \frac{x_3}{x_1},
            \dots,
            \frac{x_n}{x_1}
        \right).
    \end{equation*}
    Η παραπάνω είναι λεία συνάρτηση γιατί \( x_1 \neq 0 \) στο \( \phi_0(U_0
    \cap U_1) \). Για κάθε άλλο \( U_i \cap U_j \) η διαδικασία είναι
    αντίστοιχη. Έτσι, η συλλογή \( \{ (U_i, \phi_i) \}_{i=0,\dots,n} \) είναι
    λείος άτλαντας για ο \( \mathbb{R}P^n \). Για να δείξουμε ότι ο \( \mathbb{R}P^n \)
    είναι πολλαπλότητα πρέπει να δείξουμε ότι ο \( \mathbb{R}P^n \) είναι
    δεύτερος αριθμήσιμος και \tl{Hausdorff} καθώς η τοπολογία πηλίκου δεν
    διατηρεί τις ιδιότητες αυτές.

    Κάνοντας χρήση των παρακάτω πορισμάτων άμεσα δείχνουμε ότι ο \(
    \mathbb{R}P^n \) είναι δεύτερος αριθμήσιμος.
    \begin{defn}
        Η σχέση ισοδυναμίας \( \sim \) σε έναν τοπολογικό χώρο \( S \) είναι
        ανοιχτή αν η προβολή \( \pi: S \to S/_{\sim} \) είναι ανοιχτή.
    \end{defn}
    \begin{cor}[Πρόταση 7.14 του \tl{Tu} \cite{tu2010introduction}]
        Η σχέση ισοδυναμίας \( \sim \) στο \( \mathbb{R}^{n+1} - \{0\} \) στον
        ορισμό του \( \mathbb{R}P^n \) είναι ανοιχτή σχέση ισοδυναμίας.
    \end{cor}
    \begin{cor}[Πρόταση 7.10 του \tl{Tu} \cite{tu2010introduction}]
        Αν \( \sim \) είναι ανοιχτή σχέση ισοδυναμίας σε έναν δεύτερο αριθμήσιμο
        χώρο \( S \), τότε ο τοπολογικός χώρος \( S/_{\sim} \) είναι δεύτερος
        αριθμήσιμος.
    \end{cor}
    Εφαρμόζοντας το τελευταίο πόρισμα προκύπτει άμεσα ότι ο \( \mathbb{R}P^n \)
    είναι δεύτερος αριθμήσιμος καθώς προφανώς ο \( \mathbb{R}^{n+1} - \{0\} \)
    είναι πολλαπλότητα.

    Για να δείξουμε ότι ο \( \mathbb{R}P^n \) είναι \tl{Hausdorff} θα
    χρησιμοποιήσουμε το παρακάτω θεώρημα.
    \begin{thrm}[Θεώρημα 7.7 του \tl{Tu} \cite{tu2010introduction}]
        Έστω ότι \( \sim \) είναι ανοιχτή σχέση ισοδυναμίας στον τοπολογικό χώρο
        \( S \). Τότε ο χώρος πηλίκου \( S/_{\sim} \) είναι \tl{Hausdorff} εάν
        και μόνο εάν \( \text{\tl{graph }}\Gamma \) του \( \sim \) είναι κλειστό
        στο \( S \times S \).
    \end{thrm}
    Θα δείξουμε ότι \( \{ (x, y): x \sim y \} = f^{-1}( \{0\} ) \) για κάποια
    συνεχή συνάρτηση \( f \). Η σχέση \( x \sim y \) σημαίνει \( x_i = t y_i \).
    Δηλαδή \( \frac{x_i}{y_j} = \frac{x_j}{y_j} \), ή ισοδύναμα \( x_iy_j -
    y_ix_j = 0 \) για κάθε \( i, j \).

    Έστω \( f: (\mathbb{R}^{n+1} - \{0\})
    \times (\mathbb{R}^{n+1} - \{0\}) \to \mathbb{R} \), που ορίζεται \( f(x, y)
    = \sum_{i \neq j} (x_iy_j - y_ix_j)^2 \). Η \( f \) προφανώς είναι συνεχής.

    Έστω \( y = t x \), τότε \( f(x, y) = \sum_{i \neq j} (x_itx_j - tx_ix_j)^2
    = 0 \).

    Αντίστοιχα, έστω \( f(x, y) = \sum_{i \neq j} (x_iy_j - y_ix_j)^2
    = 0 \), τότε \( x_iy_j - y_ix_j = 0 \) για κάθε \( i,j \). Επειδή \( x \neq
    0 \) θα πρέπει να υπάρχει κάποιο \( i = k \) τέτοιο ώστε \( x_k \neq 0 \).
    Έτσι \( y_j = \frac{y_k}{x_k}x_j \) για κάθε \( j \) και επομένως \( y = tx
    \).

    Έτσι \( f^{-1}( \{0\} ) = \{ (x,y): x \sim y \} \) που συμπίπτει με το
    \tl{graph} \(\Gamma\). Επειδή η \( f \) είναι
    συνεχής και το \( \{0\} \) είναι κλειστό στο \( \mathbb{R} \),
    \( \{ (x,y): x \sim y \} \) είναι κλειστό στο \( (\mathbb{R}^{n+1} - \{0\})
    \times (\mathbb{R}^{n+1} - \{0\}) \). Τελικά ο \( \mathbb{R}P^n =
    (\mathbb{R}^{n+1} - \{0\})/_{\sim} \) είναι \tl{Hausdorff}.

    Η διαδικασία για να ορίσουμε διανυσματικό πεδίο θα γίνει σύμφωνα με το βιβλίο
    του \tl{Lee} \cite{lee2012introduction},
    και θα χρησιμοποιήσουμε τα παρακάτω.
    \begin{defn}
        Έστω \( F: M \to N \) λεία και \( X \) είναι διανυσματικό πεδίο στο \( M
        \), και έστω ότι υπάρχει διανυσματικό πεδίο \( Y \) στο \( N \) με την
        ιδιότητα ότι για κάθε \( p \in M \), \( dF_p(X_p) = Y_{F(p)} \), όπου
        \( dF_p: T_pM \to T_{F(p)}N \) είναι το διαφορικό της \( F \) στο \( p \).
        Τότε λέμε ότι τα διανυσματικά πεδία \( X \) και \( Y \) είναι
        \tl{F}-συσχετισμένα (\tl{F-related}).
    \end{defn}
    \begin{prop}[Πρόταση 8.19 του \tl{Lee} \cite{lee2012introduction}]
        Έστω \( M \) και \( N \) είναι λεία διανυσματικά πεδία και \( F: M \to N
        \) είναι διαφορομορφισμός. Για κάθε \( X \in \mathfrak{X}(M) \), υπάρχει
        μοναδικό διανυσματικό πεδίο στο \( N \) που είναι \tl{F}-συσχετισμένο με
        το \( X \).
    \end{prop}
    \begin{prop}[Πρόταση 8.11 του \tl{Tu} \cite{tu2010introduction}]
        Έστω \( F: M \to N \) μία λεία απεικόνιση μεταξύ πολλαπλοτήτων, \( p \in
        M \) και έστω \( (U, x_1, \dots, x_m), (V, y_1, \dots, y_n) \) τοπικοί
        χάρτες στα σημεία \( p \in M \) και \( F(p) \in N \) αντίστοιχα. Τότε,
        ως προς τις βάσεις \( \{\partial/ \partial x_j |p \} \) του \( T_pM \)
        και \( \{\partial/ \partial y_i |F(p) \} \) του \( T_{F(p)}N \), το
        διαφορικό \( dF_p: T_pM \to T_{F(p)}N \) παρίσταται από τον πίνακα \(
        \left( \partial F_i/ \partial x_j(p) \right) \), όπου \( F_i = y_i \circ
        F \) η \(i\)-συνιστώσα της \( F \).
    \end{prop}
    Με τα παραπάνω βλέπουμε ότι εύκολα μπορούμε να ορίσουμε διανυσματικό πεδίο
    στον προβολικό χώρο. Δημιουργώντας ένα διανυσματικό πεδίο στο χώρο
    \( \mathbb{R}^{n+1} - \{0\} \) μπορούμε να μεταφέρουμε το πεδίο μέσω της
    απεικόνισης \( \pi: \mathbb{R}^{n+1} - \{0\} \to \mathbb{R}P^n \) στον
    πραγματικό προβολικό χώρο.
\end{solution}

\begin{exercise}{2015/16 2β}
    Δείξτε ότι κάθε τετραγωνικός πίνακας \( A \) είναι όμοιος με το άθροισμα
    πινάκων \( S \) και \( N \), όπου \( S \) είναι διαγώνιος και \( N \) είναι
    μηδενοδύναμος (Υπόδειξη: προκύπτει από τη μορφή \tl{Jordan}). Επομένως,
    δείξτε ότι η εκθετική συνάρτηση του \( A \) είναι όμοια με την \(
    e^{St}e^{Nt} \) και δώστε τη γενική μορφή των δύο αυτών εκθετικών συναρτήσεων
    και επομένως της \( e^{At} \).
\end{exercise}
\begin{solution}{2015/16 2β}
    Αν \( A \in M_n(\mathbb{C}) \) τότε από τη μορφή \tl{Jordan}, υπάρχει
    αντιστρέψιμος πίνακας \( P \) που αποτελείται από τα γενικευμένα
    ιδιοδιανύσματα του \( A \), τέτοιος ώστε να ισχύει \( A  = PJP^{-1}
    \) και ο \( J \) να έχει τη μορφή
    \begin{center}
        \(
            J =
            \left(
                \begin{array}{*4{c}}
                    A_{\lambda_1} &               &        & 0\\
                    & A_{\lambda_2} &        &  \\
                    &               & \ddots &  \\
                    0          &               &        & A_{\lambda_k}
                \end{array}
            \right),
        \)
    \end{center}
    όπου κάθε μπλοκ είναι της μορφής
    \begin{center}
        \(
            A_{\lambda_i} =
            \left(
                \begin{array}{*5{c}}
                    \lambda_i     &     1         &    0      &         &  0        \\
                    & \lambda_i     &    1      &         &           \\
                    &               & \lambda_i &         &           \\
                    &               &           &  \ddots &           \\
                    0          &               &           &         & \lambda_i  \\
                \end{array}
            \right).
        \)
    \end{center}
    Επομένως ο πίνακας \( J \) μπορεί να γραφτεί
    \begin{equation*}
        J = D + U
    \end{equation*}
    όπου \( D \) είναι διαγώνιος πίνακας με τις ιδιοτιμές του \( A \) και
    \( U \) είναι ένας αυστηρώς άνω τριγωνικός πίνακας που είναι μηδενοδύναμος
    στην κανονική του μορφή. Επομένως, άμεσα έχουμε ότι
    \begin{equation*}
        A = P(D + U)P^{-1}.
    \end{equation*}
    Δηλαδή ο \( A \) είναι όμοιος με ένα διαγώνιο πίνακα και έναν μηδενοδύναμο.
    Επίσης, για τον
    \( A \) ισχύει
    \begin{equation*}
        A = PDP^{-1} + PUP^{-1},
    \end{equation*}
    ή ισοδύναμα
    \begin{equation*}
        A = S + N.
    \end{equation*}
    Ο \(S\) είναι διαγωνοποιήσιμος πίνακας, που είναι
    όμοιος με το διαγώνιο πίνακα \(D\). Επίσης, ο \(N\) είναι όμοιος με την
    κανονική μορφή του μηδενοδύναμου \(U\), και άρα ο \( N \) είναι μηδενοδύναμος.

    Το παρακάτω θεώρημα αποδεικνύεται στο παράρτημα \tl{III} του
    \cite{hirsch1974differential}.
    \begin{thrm}[Θεώρημα 1, σ.110
        των \tl{Hirsch} και \tl{Smale} \cite{hirsch1974differential}]
        Έστω \(T\) τελεστής στο \(E\), όπου \(E\) μιγαδικός διανυσματικός χώρος,
        ή ο \(E\) είναι πραγματικός αν ο \(T\) έχει πραγματικές ιδιοτιμές. Τότε
        ο \(E\) είναι το ευθύ άθροισμα των γενικευμένων ιδιοδιανυσμάτων του
        \(T\). Η διάσταση κάθε γενικευμένου ιδιοχώρου ισούται με την
        πολλαπλότητα της αντίστοιχης ιδιοτιμής.
    \end{thrm}
    Από το παραπάνω θεώρημα προκύπτει ότι \(A = A_1 \oplus \dots \oplus A_r\).
    Κάθε \( A_k \) αντιστοιχεί στην ιδιοτιμή \( \lambda_k \), άρα ισχύει
    \begin{equation*}
        A_k = S_k + N_k,
    \end{equation*}
    με \( S_k = \lambda_k I \) και προφανώς ισχύει \( S_kN_k = N_kS_k \).
    Επίσης, ισχύει \(S = S_1 \oplus \dots \oplus S_r\) και
    \(N = N_1 \oplus \dots \oplus N_r\), και άρα ισχύει \( SN = NS \).
    Έτσι από τη σχέση \eqref{eq:1415_ex_4_etab} προκύπτει για τις εκθετικές
    συναρτήσεις
    \begin{equation*}
        e^{St}e^{Nt} = e^{t(S + N)} = e^{tA}.
    \end{equation*}
\end{solution}
\begin{exercise}{2015/16 3}
    Γράψτε προσεκτικά την απόδειξη του θεωρήματος του \tl{Frobenius}, βασισμένη
    στον τρόπο που δώσαμε στο μάθημα.
\end{exercise}
\begin{solution}{2015/16 3}
    Η απόδειξη του θεωρήματος του \tl{Frobenius} είναι του \tl{Lundell}
    \cite{lundellfrobenius}.
    \begin{thrm}
        Μία (λεία) \(r-\)κατανομή \( D \) (σταθερού βαθμού ή διάστασης) σε μία
        \(m-\)πολλαπλότητα \( M \) είναι ενειλιγμένη εάν και μόνο εάν η \( D \)
        είναι πλήρως ολοκληρώσιμη.
    \end{thrm}
    Υπενθυμίζουμε ότι \( D \) είναι ενειλιγμένη αν σε μία γειτονία κάθε
    σημείου της \( M \) υπάρχει τοπική βάση διανυσματικών πεδίων \( \{ X_1, X_2,
    \dots, X_r \} \) τέτοια ώστε
    \begin{equation*}
        [X_i, X_j] = \sum_{k=1}^r c^k_{i,j} X_k \quad \text{\gr{για }} 1 \leq i,
        j \leq r,
    \end{equation*}
    και \( D \) είναι πλήρως ολοκληρώσιμη για κάθε σημείο \( p \in M \) υπάρχει
    τοπικό σύστημα συντεταγμένων \( u: (U, p) \to (\mathbb{R}^m, 0) \) τέτοιο
    ώστε \( \{ \partial/\partial u^1, \dotsm \partial/ \partial u^r \} \) να
    αποτελούν μία τοπική βάση για τη \( D \).

    Επίσης, αν \( D \) είναι πλήρως ολοκληρώσιμη τότε είναι ενειλιγμένη.
    Επομένως θα υποθέσουμε ότι αν η \( D \) είναι ενειλιγμένη τότε θα δείξουμε ότι
    αν είναι πλήρως ολοκληρώσιμη.
    \begin{lemma}[Λήμμα 1 του \tl{Lundell} \cite{lundellfrobenius}]
        Υπάρχει τοπικό σύστημα συντεταγμένων \( v:(V, p) \to (\mathbb{R}^m, 0)
        \) τέτοιο ώστε η \( D \) να έχει τοπική βάση \( \{ X_1, \dots, X_r \} \)
        της μορφής
        \begin{equation*}
            X_i = \frac{\partial}{\partial v^i} + \sum_{j= r+1}^m b^j_i
            \frac{\partial}{\partial v^j} \quad \text{\gr{για }} 1 \leq i \leq
            r.
        \end{equation*}
    \end{lemma}
    \begin{proof}
        Επιλέγουμε τοπικές συντεταγμένες \( v:(V', p) \to (\mathbb{R}^m, 0) \) και
        έστω \( \{ Y_1, \dots, Y_r \} \) να είναι τοπική βάση των
        διανυσματικών πεδίων της \( D \) στο \( V' \). Τότε
        \begin{equation*}
            Y_i = \sum_{j=1}^m a_i^j \frac{\partial}{\partial v^j} \quad
            \text{\gr{για }} 1 \leq i\leq r,
        \end{equation*}
        όπου \( a^j_i \) είναι \( C^{\infty} \) συναρτήσεις στο \( V' \).
        Καθώς \( Y_i \) είναι γραμμικώς ανεξάρτητα, ο \( r \times m \)
        πίνακας \( A = (a_i^J) \) είναι βαθμού \( r \), και χωρίς να
        χάσουμε τη γενικότητα, μπορούμε να υποθέσουμε ότι σε κάποια
        (πιθανόν μικρότερη) γειτονιά \( V \) του \( p \) ο \( r \times r \)
        υποπίνακας \( A' = (a_i^j), 1 \leq i, j \leq r \), είναι μη ιδιάζων.
        Αν \( A'^{-1} = (\hat{a}^j_i) \), θέτουμε \( X_i = \sum_{j=1}^r
        \hat{a}_i^j Y_j \).
    \end{proof}
    \begin{cor}[Πόρισμα του \tl{Lundell} \cite{lundellfrobenius}]
        Αν η \(r-\)κατανομή \(D\) είναι ενειλιγμένη, τότε η τοπική βάση του
        προηγούμενου λήμματος ικανοποιεί \( [X_i, X_j]=0 \) για \( 1 \leq i, j
        \leq r\).
    \end{cor}
    \begin{proof}
        Παρατηρούμε ότι
        \begin{equation*}
            [X_i, X_j] \in \text{\tl{span}} \left\{ \frac{\partial}{\partial
            v^{r+1}}, \dots, \frac{\partial}{\partial v^m} \right\} \quad
            \text{\gr{για }} 1 \leq i, j \leq r.
        \end{equation*}
        Από την άλλη, καθώς η \( D \) είναι ενειλιγμένη, \( [X_i, X_j] \in
        \text{\tl{span}}\{X_1, \dots, X_r \} \) για \( 1 \leq i, j \leq r \).
        Έτσι
        \begin{equation*}
            [X_i, X_j] \in \text{\tl{span}}\{X_1, \dots, X_r \}
            \cap \text{\tl{span}}
            \left\{ \frac{\partial}{\partial v^{r+1}}, \dots,
            \frac{\partial}{\partial v^m} \right\}
            = \{0\}.
        \end{equation*}
    \end{proof}
    \begin{lemma}[Πρόταση 1.53 του \tl{Warner}
        \cite{warner2013foundations}]\label{lemma_flow_box}
        Αν \( X \) είναι διανυσματικό πεδίο με \( X(p) \neq 0 \), τότε υπάρχει
        σύστημα συντεταγμένων \( u: (U, p) \to (\mathbb{R}^m, 0) \) τέτοιο ώστε
        στο \( U \), \( X = \partial/ \partial u^1 \).
    \end{lemma}
    Η απόδειξη θα γίνει με επαγωγή στο \( r \). Η περίπτωση που το \( r = 1 \)
    είναι το παραπάνω λήμμα και η απόδειξη αυτού βασίζεται στο θεώρημα ύπαρξης
    και μοναδικότητας των διαφορικών εξισώσεων. Έστω ότι \( r \geq 2 \). Για να
    εφαρμόσουμε την επαγωγή υποθέτουμε ότι ισχύει για \( r - 1 \).
    Χρησιμοποιώντας το παραπάνω πόρισμα, υποθέτουμε ότι σε μία γειτονιά του
    \( p \) η κατανομή \( D \) έχει βάση \( \{ X_1, \dots, X_r \} \) που
    ικανοποιεί \( [X_i, X_j] = 0 \). Με βάση την επαγωγική υπόθεση, επιλέγουμε
    σύστημα συντεταγμένων \( v:(V, p) \to (\mathbb{R}^m, 0) \) έτσι ώστε στο \(
    V \), έχουμε
    \begin{equation*}
        \text{\tl{span}}\{X_1, \dots, X_{r-1} \} =
        \text{\tl{span}}
        \left\{ \frac{\partial}{\partial v^{1}}, \dots, \frac{\partial}{\partial
        v^{r-1}} \right\}.
    \end{equation*}
    Καθώς τα \( \partial/ \partial v^i \) και \( X_i \) σχετίζονται από μη
    ιδιάζων γραμμικό μετασχηματισμό και \( [X_i, X_r] = 0 \), βλέπουμε ότι
    \begin{align*}
        \left[ \frac{\partial}{\partial v^i}, X_r \right] &\in
        \text{\tl{span}} \{ X_1, \dots, X_{r-1} \} \\
        &= \text{\tl{span}}
        \left\{ \frac{\partial}{\partial v^{1}}, \dots, \frac{\partial}{\partial
        v^{r-1}} \right\} \quad \text{\gr{για }} 1 \leq i \leq r-1.
    \end{align*}
    Σε μία γειτονιά του \( p \) μπορούμε να γράψουμε \( X_r = \sum_{i=1}^m a^i
    \partial / \partial v^i \), και η συνθήκη στην αγκύλη δίνει
    \begin{equation*}
        \frac{\partial a^i}{\partial v^j} = 0 \quad \text{\gr{για }} r \leq i
        \leq m \text{\gr{ και }} 1 \leq j \leq r-1.
    \end{equation*}
    Έτσι για \( r \leq i \leq m \), τα \( a^i \) είναι ανεξάρτητα των \( v^1,
    \dots, v^{r-1} \). Αν θέσουμε
    \begin{equation*}
        Y = X_r - \sum_{i=1}^{r-1} a^i \frac{\partial}{\partial v^i} =
        \sum_{i=r}^m a^i \frac{\partial}{\partial v^i},
    \end{equation*}
    τότε
    \begin{equation*}
        \text{\tl{span}} \{ X_1, \dots, X_{r} \} \\
        = \text{\tl{span}}
        \left\{ \frac{\partial}{\partial v^{1}}, \dots, \frac{\partial}{\partial
        v^{r-1}}, Y \right\}
    \end{equation*}
    και το \( Y \) εξαρτάται μόνο από τα \( v^r, \dots, v^m \). Χρησιμοποιώντας
    το λήμμα \ref{lemma_flow_box}, μπορούμε να επιλέξουμε σύστημα συντεταγμένων
    \( u:(U, p) \to (\mathbb{R}^m, 0) \) της μορφής
    \begin{equation*}
        u^i =
        \begin{cases}
            v^i \quad \text{\gr{για }} 1 \leq i \leq r-1, \\
            \tilde{u}^i(v^r, \dots, v^m) \quad \text{\gr{για }} r \leq i \leq m \\
        \end{cases}
    \end{equation*}
    τέτοιο ώστε
    \begin{align*}
        \frac{\partial}{\partial u^i} &= \frac{\partial}{\partial v^i} \quad
        \text{\gr{για }} 1 \leq i \leq r-1, \\
        \frac{\partial}{\partial u^r} &= \frac{\partial}{\partial v^r}
        + \sum_{k=r+1}^m a^k \frac{\partial}{\partial v^k} = Y.
    \end{align*}
    Έχουμε \( \{\partial/ \partial u^1, \dots, \partial/ \partial u^r \} \)
    είναι βάση για τη \( D \) στο \( U \).
\end{solution}

\begin{exercise}{2015/16 4}
    \emph{Επιχειρήματα γενικότητας} (δηλαδή θα υποθέτετε, κάθε φορά όπου
    χρειάζεται, ότι η τιμή της συνάρτησης που θεωρείται είναι \emph{κανονική}.
    Οι διατυπώσεις περιλαμβάνουν λοιπόν εκφράσεις όπως \tl{``}γενικά\tl{''},
    \tl{``}τυπικά\tl{''} κλπ με ταυτόσημη ερμηνεία.)
    \begin{enumerate}
        \item Τι σύνολο είναι, γενικά, το σύνολο των σημείων όπου δύο τυπικά
            διανυσματικά πεδία στο χώρο \( \mathbb{R}^3 \) είναι γραμμικά
            εξαρτημένα;
    \end{enumerate}
\end{exercise}
\begin{solution}{2015/16 4}
    Διανυσματικό πεδίο σε μία λεία πολλαπλότητα \( M \) είναι μία τομή της
    απεικόνισης \( \pi: TM \to M \). Πιο συγκεκριμένα, ένα διανυσματικό πεδίο
    είναι μία συνεχής απεικόνιση \( X: M \to TM \), που συνήθως γράφεται \( p
    \mapsto X_p \), με την ιδιότητα \( \pi \circ X = \text{\tl{Id}}_M \), ή
    ισοδύναμα, \( X_p \in T_pM \) για κάθε \( p \in M \). Σε τοπικό σύστημα
    συντεταγμένων \( (U, \phi) = (U,x_1, \dots, x_n ) \) στην \( M \), η τιμή
    του διανυσματικού πεδίου \( X \) στο \( p \in U \) είναι ένας γραμμικός
    συνδυασμός
    \begin{equation*}
        X(p) = \sum_{i=1}^n X^i(p) \frac{\partial}{\partial x_i}.
    \end{equation*}
    Τα διανυσματικά πεδία \( X_1, \dots, X_m \) με \( m \leq n \) είναι
    γραμμικώς ανεξάρτητα στο \( V \subset M^n \) αν κάθε ένα από τα
    \( X_1(p), \dots, X_m(p) \) είναι γραμμικώς ανεξάρτητα διανύσματα για
    κάθε \( p \in V \).

    Επομένως αν \( X, Y \)  είναι διανυσματικά πεδία στο \( \mathbb{R}^3 \)
    τότε πρέπει οι συνιστώσες των διανυσματικών πεδίων σε τοπικές συντεταγμένες,
    να είναι γραμμικώς εξαρτημένες για κάθε σημείο του \( \mathbb{R}^3 \), ή
    ισοδύναμα να ισχύει η σχέση \( X(p) \times Y(p) = 0 \) για κάθε \( p \in
    \mathbb{R}^3 \). Γεωμετρικά, μπορεί να ερμηνευτεί ότι τουλάχιστον ένα διάνυσμα που
    παράγεται από το \( X \) είναι παράλληλο με τουλάχιστον ένα διάνυσμα που παράγεται
    από το \( Y \) σε κάθε σημείο που ορίζονται τα πεδία.
\end{solution}

\begin{exercise}{2016/17 1} Έστω \( A \in M_n(\mathbb{C}) \) τότε
    \begin{enumerate}[label=(\alph*)]
        \item Δείξτε το θεώρημα \tl{Cayley–Hamilton} με βάση τη μορφή \tl{Jordan}.
        \item Βρείτε \( 1-2 \) παραδείγματα που το ελάχιστο πολυώνυμο και το
            χαρακτηριστικό πολυώνυμο διαφέρουν.
        \item Δείξτε ότι \( \det e^A = e^{\text{\tl{tr}}A} \).
    \end{enumerate}
\end{exercise}
\begin{solution}{2016/17 1}
    (α) Από τη μορφή \tl{Jordan} ισχύει \( A = PJP^{-1}\). Έστω οι
    ιδιοτιμές του \( A \) και άρα και του \( J \), \( \lambda_1, \dots,
    \lambda_k \) με τις αντίστοιχες αλγεβρικές πολλαπλότητες \( n_1, \dots, n_k
    \). Το χαρακτηριστικό πολυώνυμο του \( A \) είναι
    \begin{equation*}
        p_A(\lambda) = \det{(\lambda I - A)} = \prod_{i=1}^k(\lambda -
        \lambda_i)^{n_i},
    \end{equation*}
    και για τον πίνακα \( A \) προκύπτει
    \begin{equation*}
        p_A(A) = \prod_{i=1}^k(A - \lambda_iI_n)^{n_i}.
    \end{equation*}
    Έστω το στοιχείο \( j \) με αλγεβρική πολλαπλότητα \( n_j \) τότε
    \begin{equation}\label{eq:1516_ex_1_a1}
        (A - \lambda_j I_n)^{n_j} = (A - \lambda_j I_n) \dots (A - \lambda_j
        I_n),
    \end{equation}
    και από τη μορφή \tl{Jordan} για ένα από τα παραπάνω στοιχεία του γινομένου
    ισχύει
    \begin{equation*}
        (A - \lambda_j I_n) = PJP^{-1} - \lambda_j I_n = PJP^{-1} - \lambda_j
        PP^{-1} = P(J - \lambda_j I_n)P^{-1},
    \end{equation*}
    και έτσι η \eqref{eq:1516_ex_1_a1} γίνεται
    \begin{align*}
        (A - \lambda_j I_n)^{n_j} &= P(J - \lambda_j I_n)P^{-1} \dots P(J -
        \lambda_j I_n)P^{-1} \\
        &=P(J - \lambda_j I_n)^{n_j}P^{-1}.
    \end{align*}
    Με την ίδια διαδικασία για κάθε στοιχείο, εύκολα φαίνεται ότι για το
    χαρακτηριστικό πολυώνυμο ισχύει
    \begin{equation*}
        p_A(A) = \prod_{i=1}^k(A - \lambda_iI_n)^{n_i} =
        P\left( \prod_{i=1}^k(J - \lambda_iI_n)^{n_i} \right) P^{-1}.
    \end{equation*}
    Όμως \( J \) και \( A \) έχουν τις ίδιες ιδιοτιμές, άρα \( (J - \lambda_i
    I_n) \) μας δίνει είτε έναν μηδενικό πίνακα είτε ένα αυστηρώς άνω τριγωνικό πίνακα,
    δηλαδή με μηδενικά στη διαγώνιο. Στην πρώτη περίπτωση ξεκάθαρα ισχύει
    \( p_A(A) = 0 \). Στη δεύτερη περίπτωση έχουμε ότι κάθε αυστηρώς τριγωνικός
    πίνακας είναι μηδενοδύναμος, όπως εύκολα φαίνεται αν τον υψώσουμε σε δυνάμεις
    και κάνουμε τους υπολογισμούς. Επομένως, \( (J - \lambda_iI_n)^{n_i} = 0 \)
    για κάποια δύναμη το πολύ ίση με \( n_i \), και έτσι
    \( \left( \prod_{i=1}^k(J - \lambda_iI_n)^{n_i} \right) \) που
    σημαίνει ότι \( p_A(A) = 0 \).

    (β) Το χαρακτηριστικό πολυώνυμο και το ελάχιστο πολυώνυμο είναι ίσα όταν η
    διάσταση κάθε \tl{Jordan} μπλοκ είναι \( 1 \), ή με άλλα λόγια όταν όλες οι
    ιδιοτιμές είναι διακριτές. Για παράδειγμα ο πίνακας
    \begin{equation*}
        A =
        \begin{pmatrix}
            0 & 1 & 0 \\
            1 & 0 & 0 \\
            0 & 0 & 1
        \end{pmatrix},
    \end{equation*}
    έχει χαρακτηριστικό πολυώνυμο
    \begin{equation*}
        p_A(\lambda) = \lambda^3 - \lambda^2 - \lambda + 1 = (\lambda +
        1)(\lambda - 1)^2,
    \end{equation*}
    και ελάχιστο πολυώνυμο
    \begin{equation*}
        m_A(\lambda) = \lambda^2 - 1 = (\lambda + 1)(\lambda - 1).
    \end{equation*}

    (γ) Από τη μορφή \tl{Jordan} έχουμε
    \begin{equation}\label{eq:1617_ex1d_det}
        \det e^A = \det e^{PJP^{-1}},
    \end{equation}
    και αναλύοντας το δεύτερο όρο έχουμε
    \begin{equation*}
        e^{PJP^{-1}} = I + PJP^{-1} + \frac{1}{2}
        \left( PJP^{-1} \right)^2 + \dots
    \end{equation*}
    αλλά για κάθε όρος εις την έστω \( k \) δύναμη ισχύει
    \begin{equation*}
        \left( PJP^{-1} \right)^k = (PJP^{-1}) \dots(PJP^{-1}) = PJ^kP^{-1}.
    \end{equation*}
    Συνεπώς, η προηγούμενη σχέση γράφεται
    \begin{align*}
        e^{PJP^{-1}} &= I + PJP^{-1} + \frac{1}{2}
        PJ^2P^{-1} + \dots \\
        &= P\left( I + J + \frac{1}{2}
        J^2 + \dots\right)P^{-1} \\
        &= Pe^{J}P^{-1}.
    \end{align*}
    Έτσι η \eqref{eq:1617_ex1d_det} γίνεται
    \begin{align*}
        \det e^A &= \det{(Pe^JP^{-1})} = \det{P} \det{e^J} \det{P^{-1}} \\
        &= \det{(PP^{-1})}\det{e^J} = \det{e^J}.
    \end{align*}
    Από τον ορισμό του εκθετικού πίνακα και από το ανάπτυγμα \tl{Taylor}
    της εκθετικής συνάρτησης προκύπτει άμεσα ότι για διαγώνιους πίνακες, άνω ή κάτω
    τριγωνικούς τα στοιχεία της διαγωνίου του αντίστοιχου εκθετικού πίνακα
    είναι \( e^{\lambda_i} \), όπου \( \lambda_i \) είναι το στοιχείο της
    διαγωνίου του πίνακα και άρα η ιδιοτιμή του.  Ο \( J \) είναι της μορφής
    \tl{Jordan} και έχει όλες τις ιδιοτιμές του στη διαγώνιο. Ο \( e^J \) είναι άνω
    τριγωνικός και οι ιδιοτιμές \( e^{\lambda_i} \) είναι στην διαγώνιο. Άρα
    \begin{equation*}
        \det{e^J} = \prod_{i=1}^n e^{\lambda_{i}} =
            e^{\sum_{i=1}^n \lambda_{i}} = e^{\text{\tl{tr }}J}.
    \end{equation*}
    Όμως ισχύει το παρακάτω.
    \begin{lemma}
        Έστω \( A, n \times m \) πίνακας και \( B, m \times n \) τότε \(
        \text{\tl{tr }}(AB) = \text{\tl{tr }}(BA) \).
    \end{lemma}
    \begin{proof}
        \begin{equation*}
            \text{\tl{tr }}(AB) =
            \sum_{i=1}^n \sum_{j=1}^m a_{ij}b_{ji} =
            \sum_{j=1}^m \sum_{i=1}^n b_{ji}a_{ij} =
            \text{\tl{tr }}(BA).
        \end{equation*}
    \end{proof}
    Άρα αφού \( J = P^{-1}AP \) με το παραπάνω ισχύει
    \begin{equation*}
        \text{\tl{tr }}(J) =
        \text{\tl{tr }}(P^{-1}AP) =
        \text{\tl{tr }}(P^{-1}PA) =
        \text{\tl{tr }}(A),
    \end{equation*}
    και συνεπώς ισχύει
    \begin{equation*}
        \det e^A = e^{\text{\tl{tr}}A}.
    \end{equation*}
\end{solution}

\begin{exercise}{2016/17 2}
    \emph{Κυκλικότητα και ελεγξιμότητα}. Αν έχουμε Διανυσματικό Χώρο \( V^n \),
    διάνυσμά του \( \vec{u} \neq \vec{0} \) και \( T: V^n \to V^n \) γραμμική,
    θεωρούμε τα διανύσματα \( \vec{u}, T\vec{u}, T^2\vec{u}, \dots \). Παράγεται
    υποχώρος το \tl{span} τους, αναλλοίωτος από τη γραμμική απεικόνιση \( T \).
    Αν \( W = \text{\tl{span}}(\vec{u}, T\vec{u}, T^2\vec{u}, \dots) \) τότε
    \( TW \subset W \). Λέμε ότι το \( \vec{u} \) είναι κυκλικό ως προς την
    \( T \) εάν \( \text{\tl{dim }}W = n \). Υπάρχουν τέτοια κυκλικά διανύσματα;

    Στο Διανυσματικό Χώρο \( M_n(\mathbb{C}) \) το σύνολο των πινάκων με
    διακριτές ιδιοτιμές \( (\lambda_i \neq \lambda_j) \) για κάθε \( i \neq j
    \), που είναι επομένως διαγωνοποιήσιμοι, είναι ανοιχτό και πυκνό στο \(
    M_n(\mathbb{C}) \). (Εδώ δώσαμε στο \( M_n(\mathbb{C}) \) κάποια μετρική
    τοπολογία). Αποδείξτε το. Επομένως, δείξτε ότι υπάρχουν κυκλικά διανύσματα
    για \tl{``}γενικό\tl{''} πίνακα \( A \). Επομένως, δείξτε ότι η συνθήκη
    γραμμικής ελεγξιμότητας του ζεύγους \( (A, b) \) ικανοποιείται γενικά,
    δηλαδή \( b, Ab, \dots, A^{n-1}b \) είναι γραμμικώς ανεξάρτητη.
\end{exercise}
\begin{solution}{2016/17 2}
    Για να δείξουμε ότι υπάρχουν τέτοια κυκλικά διανύσματα θα χρησιμοποιήσουμε
    κάποια θεωρήματα από το βιβλίο του \tl{Friedberg} \cite{friedberg2003linear}
    και για καλύτερη κατανόηση θα παραθέσουμε και τις αποδείξεις αυτών.
    \begin{thrm}[Θεώρημα 1.7 του \tl{Friedberg} \cite{friedberg2003linear}]
        \label{theo:1617_span}
        Έστω \( S \) γραμμικώς ανεξάρτητο υποσύνολο του διανυσματικού χώρου
        \( V \), και έστω \( v \) διάνυσμα του \( V \) που δεν ανήκει στο \( S \).
        Τότε \( S \cup \{v\} \) είναι γραμμικώς εξαρτημένο αν και μόνο αν \( v
        \in \text{\tl{span}}(S) \).
    \end{thrm}
    \begin{proof}
        Αν \( S \cup \{v\} \) είναι γραμμικώς εξαρτημένο, τότε υπάρχουν
        διανύσματα \( v_1, v_2, \dots, v_n \) στο \( S \cup \{v\} \) τέτοια ώστε
        \( a_1v_1 + a_2v_2 + \dots + a_nv_n = 0 \) για κάποια μη μηδενικά
        βαθμωτά \( a_1, a_2, \dots, a_n \). Επειδή \( S \) είναι γραμμικώς
        ανεξάρτητο, κάποιο από τα \( v_i \), έστω το \( v_1 \), ισούται με \( v
        \). Έτσι \( a_1v + a_2v_2 + \dots + a_nv_n = 0 \), και έτσι
        \begin{equation*}
            v  = a_1^{-1}(-a_2v_2 - \dots - a_nv_n) = - (a_1^{-1}a_2)v_2 - \dots -
            (a_1^{-1}a_n)v_n.
        \end{equation*}
        Συνεπώς \( v \) είναι γραμμικώς συνδυασμός των \( v_2, \dots, v_n \),
        που είναι στο \( S \), και έτσι έχουμε \( v \in \text{\tl{span}}(S) \).

        Αντίστροφα, έστω \( v \in \text{\tl{span}}(S) \). Τότε υπάρχουν
        διανύσματα \( v_1, v_2, \dots, v_m \) στο \( S \) και βαθμωτά \( b_1,
        b_2, \dots, b_m \) τέτοια ώστε \( v = b_1v_1 + b_2v_2 + \dots + b_mv_m
        \). Έτσι
        \begin{equation*}
            0 = b_1v_1 + b_2v_2 + \dots + b_mv_m + (-1)v.
        \end{equation*}
        Όμως \( v \neq v_i \) για \( i = 1,2,\dots,m \) και ο συντελεστής του \(
        v \) είναι μη μηδενικός στον παραπάνω γραμμικό συνδυασμό, προκύπτει ότι
        το σύνολο \( \{v_1, v_2, \dots, v_m, v \} \) είναι γραμμικώς εξαρτημένο
        και άρα \( S \cup \{v\} \) είναι γραμμικώς εξαρτημένο.
    \end{proof}
    \begin{thrm}[Θεώρημα 5.22 του \tl{Friedberg} \cite{friedberg2003linear}]
        \label{theo:1617_base}
        Έστω \( T \) γραμμικός τελεστής σε πεπερασμένης διάστασης διανυσματικό
        χώρο \( V \), και έστω \( W \) ο \( T-\)κυκλικός υποχώρος του \( V \)
        που παράγεται από μη μηδενικό διάνυσμα \( v \in V \). Έστω \( k =
        \text{\tl{dim}}(W) \). Τότε \( \{v, T(v), T^2(v), \dots, T^{k-1}(v) \}
        \) είναι βάση για το \( W \).
    \end{thrm}
    \begin{proof}
        Καθώς \( v \neq 0 \), το σύνολο \( \{v\} \)
        είναι γραμμικώς ανεξάρτητο.  Έστω \( j \) ο μεγαλύτερος θετικός ακέραιος
        τέτοιος ώστε το σύνολο
        \begin{equation*}
            \beta = \{ v, T(v), \dots, T^{j-1}(v) \}
        \end{equation*}
        να είναι γραμμικώς ανεξάρτητο. Τέτοιος ακέραιος \( j \) πρέπει να
        υπάρχει διότι \( V \) είναι πεπερασμένης διάστασης. Έστω \( Z =
        \text{\tl{span}}(\beta) \). Τότε \( \beta \) είναι βάση για το \( Z \).
        Επίσης, από το προηγούμενο θεώρημα \ref{theo:1617_span}, έχουμε \( T^j(v) \in Z \).
        Χρησιμοποιούμε την πληροφορία αυτή για να δείξουμε ότι \( Z \) είναι
        \(T-\)αναλλοίωτος υποχώρος του \( V \). Έστω \( w \in Z \). Τότε \( w \)
        είναι γραμμικώς συνδυασμός των διανυσμάτων του \( \beta \), και υπάρχουν
        βαθμωτά \( b_o, b_1, \dots, b_{j-1} \) τέτοια ώστε
        \begin{equation*}
            w = b_0v + b_1T(v) + \dots + b_{j-1}T^{j-1}(v),
        \end{equation*}
        και αν εφαρμόσουμε την απεικόνιση \( T \) στην παραπάνω
        \begin{equation*}
            T(w) = b_0T(v) + b_1T^2(v) + \dots + b_{j-1}T^{j}(v).
        \end{equation*}
        Άρα \( T(w) \) είναι γραμμικώς συνδυασμός των διανυσμάτων του \( Z \)
        και συνεπώς ανήκει στο \( Z \). Έτσι το \( Z \) είναι \(T-\)αναλλοίωτο.
        Επίσης, \( v \in Z \) και \( W \) είναι ο μικρότερος \(T-\)αναλλοίωτος
        υποχώρος του \( V \) που περιέχει το \( v \), δηλαδή κάθε \(T-\)αναλλοίωτος
        υποχώρος του \( V \) που περιέχει το \( v \) περιέχει και το \( W \),
        έτσι \( W \subset Z \).  Όμως \( Z \subset W \), και έτσι καταλήγουμε ότι
        \( Z = W \). Επομένως \( \beta \) είναι βάση για το \( W \) και έτσι
        \( \text{\tl{dim}}(W) = j \), και άρα \( j = k \).
    \end{proof}
    Αφού παραθέσαμε τα απαραίτητα θεωρήματα, το να δείξουμε ότι υπάρχουν κυκλικά
    διανύσματα θα γίνει σε δύο στάδια, μέσω των παρακάτω.

    \begin{lemma}\label{lemma:1617_eigenvec}
        Έστω \( T \) γραμμικός τελεστής σε πεπερασμένης διάστασης διανυσματικό χώρο
        \( V \), και έστω \( W \), \(T-\)αναλλοίωτος υποχώρος του \( V \). Έστω \(
        v_1, v_2, \dots, v_k \) ιδιοδιανύσματα του \( T \) που αντιστοιχούν σε
        διακριτές ιδιοτιμές. Αν \( v_1 + v_2 + \dots + v_k \) ανήκει στο \( W \),
        τότε \( v_i \in W \) για κάθε \( i \).
    \end{lemma}
    \begin{proof}
        Η απόδειξη θα γίνει με επαγωγή στο \( k \). Για \( k = 1 \), προφανώς \(
        v_1 \in W \). Αν η πρόταση ισχύει για \( k = m-1 \), θα εξετάσουμε την
        περίπτωση που \( k = m \). Αν έχουμε \( v = v_1 + v_2 + \dots + v_m \in
        W \), επειδή είναι \(T-\)αναλλοίωτος ισχύει
        \begin{equation*}
            T(v) = \lambda_1v_1 + \lambda_2v_2 + \dots + \lambda_mv_m \in W,
        \end{equation*}
        όπου \( \lambda_i \) είναι οι διακριτές ιδιοτιμές. Αλλά \( v \in W \)
        άρα και \( \lambda_mv \in W \). Επομένως
        \begin{equation*}
            T(v) -\lambda_mv = (\lambda_1 - \lambda_m)v_1 + (\lambda_1 - \lambda_m)v_2 + \dots +
            (\lambda_1 - \lambda_m)v_{m-1} \in W.
        \end{equation*}
        Επειδή οι ιδιοτιμές είναι διακριτές οι παραπάνω διαφορές είναι μη
        μηδενικές. Αν εφαρμόσουμε την επαγωγική υπόθεση τότε έχουμε
        \begin{equation*}
            (\lambda_1 - \lambda_m)v_1, (\lambda_1 - \lambda_m)v_2, \dots ,
            (\lambda_1 - \lambda_m)v_{m-1} \in W,
        \end{equation*}
        και άρα \( v_1, v_2, \dots, v_{m-1} \in W \). Όμως έχουμε \( v = v_1 + v_2 + \dots + v_m \in
        W \) άρα ισχύει \( v_m = v - v_1 - v_2 - \dots - v_{m-1} \in W \).
    \end{proof}
    \begin{lemma}\label{lemma:1617_cyclic}
        Έστω \( T \) γραμμικός τελεστής σε \(n\) διαστάσεων διανυσματικό χώρο
        \( V \), τέτοιος ώστε \( T \) να έχει \( n \) διακριτές ιδιοτιμές. Τότε
        ο \( V \) είναι \(T-\)κυκλικός υποχώρος του εαυτού του.
    \end{lemma}
    \begin{proof}
        Έστω \( \{v_1, v_2, \dots, v_n\} \) τα ιδιοδιανύσματα που αντιστοιχούν
        στις \( n \) διακριτές ιδιοτιμές. Έστω
        \begin{equation*}
            v = v_1 + v_2 + \dots + v_n,
        \end{equation*}
        και έστω \( W \) ο \(T-\)κυκλικός υποχώρος που παράγεται από το \( v \).
        Από το προηγούμενο λήμμα \ref{lemma:1617_eigenvec} όμως, ο \( W \) πρέπει να περιέχει όλα τα
        ιδιοδιανύσματα \( v_i \). Αλλά αυτό σημαίνει ότι η διάσταση του \( W \)
        είναι \( n \) και το σύνολο από το θεώρημα \ref{theo:1617_base}
        \begin{equation}\label{eq:1617_cyclic_base}
            \{ v, T(v), T^2(v), \dots, T^{n-1}(v) \}
        \end{equation}
        είναι βάση για το \( W \) και άρα για το διανυσματικό χώρο \( V \).
    \end{proof}
    Άρα σύμφωνα με τα παραπάνω κυκλικά διανύσματα ως προς την \( T \) υπάρχουν
    όταν η απεικόνιση \(T\) έχει διακριτές ιδιοτιμές.

    Ένα σύνολο \( U \subset M \), όπου το \( M \) είναι εφοδιασμένο με κάποια
    μετρική, είναι ανοιχτό αν για κάθε \( X \in U \) υπάρχει ανοιχτή μπάλα γύρω
    από το \( X \) που περιέχεται στο \( U \), όπου για κάποιο \( a > 0 \) η
    ανοιχτή μπάλα γύρω από το \( X \) ακτίνας \( a \),
    \begin{equation*}
        \{ Y \in M: \| Y - X \| < a \}
    \end{equation*}
    περιέχεται στο \( U \). Γεωμετρικά αυτό μπορεί να ερμηνευτεί, για κάθε
    \( X \) που ανήκει σε ένα ανοιχτό σύνολο \( U \), κάθε σημείο επαρκώς κοντά
    στο \( X \) ανήκει στο \( U \).

    Ένα σύνολο \( U \subset M \), όπου το \( M \) είναι εφοδιασμένο με κάποια
    μετρική, είναι πυκνό αν υπάρχουν σημεία στο \( U \) αυθαίρετα κοντά σε κάθε
    σημείο του \( M \). Πιο συγκεκριμένα, αν \( X \in M \) τότε για κάθε \(
    \epsilon > 0 \) υπάρχει κάποιο \( Y \in U \) με \( \| X - Y \| < \epsilon
    \). Ισοδύναμα χωρίς την ανάγκη μετρικής, \( U \) είναι πυκνό στο \( M \) αν
    \( V \cap U \) είναι μη κενό για κάθε μη κενό σύνολο \( V \in M \).

    Ενδιαφέρον παρουσιάζουν τα σύνολο που είναι ανοιχτά και πυκνά. Ένα τέτοιο σύνολο
    \( U \) χαρακτηρίζεται: Κάθε σημείο στο συμπληρωματικό του \( U \) μπορεί να
    προσεγγιστεί αυθαίρετα κοντά από σημεία του \( U \) (καθώς \( U \) είναι
    πυκνό), αλλά κανένα σημείο στο \( U \) μπορεί να προσεγγιστεί αυθαίρετα
    κοντά από σημεία του συμπληρωματικού του (γιατί \( U \) είναι ανοιχτό).

    Αρχικά θα δείξουμε ότι το σύνολο των πινάκων στο διανυσματικό χώρο
    \( M_n(\mathbb{C}) \) με διακριτές ιδιοτιμές είναι πυκνό στο \( M_n(\mathbb{C}) \).
    Δηλαδή, αν \( A \in M_n(\mathbb{C}) \) με διακριτές ιδιοτιμές και
    \( \| \cdot \| \) κάποια νόρμα στο  \( M_n(\mathbb{C}) \), τότε για κάποιο
    πραγματικό αριθμό \( \epsilon > 0\) υπάρχει πίνακας \( A_1 \in M_n(\mathbb{C}) \)
    που έχει διακριτές ιδιοτιμές και ισχύει \( \| A - A_1 \| < \epsilon \).
    Δηλαδή για κάθε πίνακα \( A \in M_n(\mathbb{C}) \) υπάρχει διαγωνοποιήσιµος
    πίνακας αυθαίρετα κοντά στον \( A \).

    Από τη μορφή \tl{Jordan} υπάρχει αντιστρέψιμος \( P \in M_n(\mathbb{C}) \)
    και \( J \in M_n(\mathbb{C}) \) μορφής \tl{Jordan} τέτοιοι ώστε \( A =
    PJP^{-1} \). Οι ιδιοτιμές του \( J \) είναι τα διαγώνια στοιχεία και έτσι μπορούμε
    να διαταράξουμε αυτά τα στοιχεία τόσο ώστε να πάρουμε έναν πίνακα
    \( J_1 \) μορφής \tl{Jordan} με διακριτές ιδιοτιμές. Μπορούμε να το κάνουμε
    με τέτοιο τρόπο ούτως ώστε \( \| J - J_1 \| < \epsilon \). Έστω \( A_1 =
    PJ_1P^{-1} \), που έχει ίδιες ιδιοτιμές με τον \( J_1 \). Όμως \( A =
    PJP^{-1} \), άρα \( A - A_1 = P(J - J_1)P^{-1} \) και τελικά \( \| A - A_1
    \| = \| J - J_1 \| < \epsilon \).

    Στη συνέχεια ότι το σύνολο των πινάκων στο διανυσματικό χώρο
    \( M_n(\mathbb{C}) \) με διακριτές ιδιοτιμές είναι ανοιχτό στο \( M_n(\mathbb{C}) \).

    Έστω \( A \in M_n(\mathbb{C}) \). Αν διαταράξουμε \tl{``}λίγο\tl{''} τα στοιχεία
    του \( A \), τότε οι συντελεστές του χαρακτηριστικού πολυωνύμου \( A \) θα
    διαταραχθούν \tl{``}λίγο\tl{''}. Έτσι και οι ρίζες του πολυωνύμου θα διαταραχθούν
    \tl{``}λίγο\tl{''}. Συνεπώς, αν ο πίνακας \( A \) έχει διακριτές ιδιοτιμές,
    πίνακες αυθαίρετα κοντά σε αυτόν, θα έχουν αυτήν την ιδιότητα και άρα το
    σύνολο με διακριτές ιδιοτιμές στο διανυσματικό χώρο \( M_n(\mathbb{C}) \) είναι ανοιχτό.

    Προηγουμένως δείξαμε ότι υπάρχουν κυκλικά διανύσματα όταν η απεικόνιση
    \( T \) έχει διακριτές ιδιοτιμές. Κάθε πίνακας είναι ουσιαστικά μία γραμμική απεικόνιση,
    άρα άμεσα προκύπτει από το \ref{lemma:1617_cyclic}, ότι για \tl{``}γενικό\tl{''}
    πίνακα υπάρχουν κυκλικά διανύσματα, καθώς αυτό σημαίνει ότι θα έχει διακριτές ιδιοτιμές.

    Επίσης, άμεσο αποτέλεσμα των προηγουμένων, είναι η συνθήκη γραμμικής
    ελεγξιμότητας του ζεύγους \( (A, b) \). Από το \ref{lemma:1617_cyclic},
    ξέρουμε ότι σύνολο \eqref{eq:1617_cyclic_base} αποτελεί βάση για τον
    αντίστοιχο χώρο που ορίζεται, δηλαδή είναι γραμμικώς ανεξάρτητο. Άρα
    για \tl{``}γενικό\tl{''} \( A \in M_n(\mathbb{C}) \) η συνθήκη
    \(b, Ab, \dots, A^{n-1}b \) αποτελεί βάση για το σύνολο με διακριτές
    ιδιοτιμές και άρα είναι γραμμικώς ανεξάρτητο. Όμως όπως δείξαμε το σύνολο
    αυτό είναι πυκνό στο \( M_n(\mathbb{C}) \) και άρα μπορούμε να πούμε ότι η
    προηγούμενη συνθήκη ικανοποιείται \tl{``}γενικά\tl{''}.
\end{solution}

\begin{exercise}{2016/17 3}
    Θυμίζουμε ότι για Σύστημα Ελέγχου \( \dot{x} = Ax + Bu \) και συνάρτηση
    ελέγχου \( u(t), t \in [0, T] \) έχει λύση
    \begin{equation*}
        x(t) = e^{At}x_0 + \sum_0^T e^{A(t - \tau)}Bu(\tau) d\tau.
    \end{equation*}
    (α) Δείξτε ότι \( e^{At} = p_0(t) + p_1(t)A + \dots + p_{n-1}(t)A^{n-1}\).
\end{exercise}
\begin{solution}{2016/17 3}
    (α) Από το θεώρημα \tl{Cayley-Hamilton} για έναν πίνακα \( A, n\times n \) ισχύει
    \begin{equation*}
        p_A(A) = A^n + c_{n-1}A^{n-1} + c_{n-2}A^{n-2} + \dots +  c_1A + c_0I = 0,
    \end{equation*}
    ή ισοδύναμα θέτοντας \( b_i = -c_i \) για \( 0 \geq i \geq n-1 \) έχουμε
    \begin{equation*}
        A^n = b_{n-1}A^{n-1} + b_{n-2}A^{n-2} + \dots + b_1A + b_0I.
    \end{equation*}
    Πολλαπλασιάζοντας την παραπάνω με \( A \) προκύπτει
    \begin{equation*}
        A^{n+1} = b_{n-1}A^{n} + b_{n-2}A^{n-1} + \dots + b_1A^2 + b_0A,
    \end{equation*}
    και με αντικατάσταση της προηγούμενης στην παραπάνω, θα πάρουμε τη σχέση του
    \( A^{n+1} \) ως προς τις \( n-1 \) δυνάμεις του \( A \). Συνεχίζοντας με
    την ίδια λογική μπορούμε να πάρουμε τη σχέση για τη δύναμη
    \( k \) ως προς τη \( n+1 \) δύναμη,
    \begin{equation*}
        A^k = \sum_{j=0}^{n-1}b_{kj}A^j.
    \end{equation*}
    Από τον ορισμό του εκθετικού πίνακα έχουμε
    \begin{align*}
        e^{At} &= \sum_{k=0}^{\infty}\frac{t^k}{k!}A^k
        = \sum_{k=0}^{\infty}\frac{t^k}{k!} \sum_{j=0}^{n-1}b_{kj}A^j \\
        &= \sum_{j=0}^{n-1} \sum_{k=0}^{\infty}\frac{t^k}{k!}b_{kj}A^j \\
        &= \sum_{j=0}^{n-1} p_j(t)A^j.
    \end{align*}
\end{solution}

\newpage
\selectlanguage{english}
\printbibliography[title=\greektext{Βιβλιογραφία}]
\end{document}
