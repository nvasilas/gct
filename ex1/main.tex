\documentclass[a4paper,11pt]{article}
\usepackage[english,greek]{babel}
\usepackage[utf8]{inputenc}
\usepackage[T1]{fontenc}
\usepackage{textcomp}

\usepackage{amsmath, amsthm, amssymb}
\usepackage{bm}

\usepackage{libertine}

\usepackage{url}
\usepackage[hidelinks,unicode]{hyperref}

\usepackage{graphicx}
\usepackage[small,bf]{caption}
\usepackage{subcaption}

\newcommand{\vc}[1]{\bm{#1}} % this is a vector
\newcommand{\mt}[1]{\boldsymbol{#1}} % this is a matrix

\newcommand{\tl}[1]{\textlatin{#1}}
\newcommand{\gr}[1]{\greektext{#1}}
\newcommand{\greek}[1]{{\selectlanguage{greek}#1}}
\newcommand{\eng}[1]{{\selectlanguage{english}#1}}

\newenvironment{exercise}[2][Άσκηση]
{\begin{trivlist}
    \item[\hskip \labelsep {\bfseries #1}\hskip \labelsep {\bfseries #2.}]}
{\end{trivlist}}

\newenvironment{solution}[2][Λύση]
{\begin{trivlist}
    \item[\hskip \labelsep {\bfseries #1}\hskip \labelsep {\bfseries #2.}]}
{\end{trivlist}}

\begin{document}

\title{Ασκήσεις Γεωμετρικής Θεωρίας Ελέγχου}
\author{Βασίλας Νικόλαος}
\maketitle
\begin{exercise}{1}
    Για τα σημεία που είναι στο βόρειο και στο ανατολικό ημισφαίριο της
    μοναδιαίας σφαίρας έχουμε δύο χάρτες $\psi_{z^+}$ και $\psi_{y^+}$. Να δείξετε
    ότι $\psi_{y^+}^{-1} \circ \psi_{z^+}$ είναι $C^k$.
\end{exercise}
\begin{solution}{1}
    Η μοναδιαία σφαίρα $S^2 \subset \mathbb{R}^3$ περιγράφεται από
    \begin{equation*}
        S^2 = \{ (x, y, z) \in \mathbb{R}^3: x^2 + y^2 + z^2 = 1\}.
    \end{equation*}
    Μπορούμε να ορίσουμε τους χάρτες $(U_{y^+}, \phi_{y^+})$ και $(U_{z^+},
    \phi_{z^+})$ όπου
    \begin{equation*}
        U_{y^+} = \{(x, y, z) \in S^2: y > 0\}, \quad \psi_{y^+}(x, y, z) = (x,
        z),
    \end{equation*}
    και
    \begin{equation*}
        U_{z^+} = \{(x, y, z) \in S^2: z > 0\}, \quad \psi_{z^+}(x, y, z) = (x,
        y).
    \end{equation*}
    Για να δείξω ότι $\psi_{z^+}$ και $\psi_{y^+}$ είναι ομοιομορφισμοί, αρκεί
    να βρω τις αντίστροφες συναρτήσεις και να δείξω ότι είναι συνεχείς. Έτσι
    έχουμε
    \begin{align*}
        \psi_{y^+}^{-1}(x, z) &= \left( x, \sqrt{1 - x^2 - z^2}, z \right) \in S^2,\\
        \psi_{z^+}^{-1}(x, y) &= \left( x, y, \sqrt{1 - x^2 - y^2} \right) \in S^2,
    \end{align*}
    όπου είναι συνεχείς συναρτήσεις και άρα ομοιομορφισμοί, αλλά και
    παραγωγίσιμες $C^{\infty}$ συναρτήσεις. Συνεπώς, αφού $\psi_{z^+}$,
    $\psi_{y^+}$ καθώς και $\psi_{z^+}^{-1}$, $\psi_{y^+}^{-1}$ είναι
    $C^{\infty}$ συναρτήσεις τότε και η σύνθεση
    $\psi_{y^+}^{-1} \circ \psi_{z^+}$ θα είναι $C^{\infty}$.
\end{solution}
\begin{exercise}{2}
    Αν $M$ είναι λεία πολλαπλότητα, $TM$ είναι η εφαπτόμενη δέσμη της $M$, $\pi:
    TM \rightarrow M$ είναι η απεικόνιση προβολής και $X: M \rightarrow TM$
    διανυσματικό πεδίο, τότε να δείξετε ότι $\pi \circ X = id_M$.
\end{exercise}
\begin{solution}{2}
    Σύμφωνα με το \footnote{\tl{Boothby, W.M., \emph{An Introduction to Differentiable
    Manifolds and Riemannian Geometry}, Academic Press, 1975}},
    ένα διανυσματικό πεδίο $X$ κλάσης $C^r$ σε μία πολλαπλότητα $M$
    είναι μία συνάρτηση η οποία σε κάθε σημείο $p \in M$ αντιστοιχεί ένα
    εφαπτόμενο διάνυσμα $X_p \in T_pM$ του οποίου οι συνιστώσες ως προς
    κάποιο τοπικό χάρτη $(U, \phi)$ είναι συναρτήσεις κλάσης $C^r$ στο πεδίο
    ορισμού $U$ των συντεταγμένων. Επομένως, $\pi \circ X = id_M$ είναι άμεση
    συνέπεια του ορισμού καθώς $(\pi \circ X)(p) = \pi(X_p) = p$ και προφανώς
    ισχύει μόνο για διανυσματικά πεδία αφού το πεδίο ορισμού της $\pi$ είναι
    $TM$. Πολλές φορές όπως για παράδειγμα στο βιβλίο \footnote{\tl{Lee, J.M., \emph{Introduction to
    Smooth Manifolds}, Springer, 2003}}, το διανυσματικό πεδίο ορίζεται ως μία συνεχής
    συνάρτηση $X: M \rightarrow TM$ με την ιδιότητα $\pi \circ X = id_M$.
\end{solution}
\begin{exercise}{3}
    Να αποδείξετε την ταυτότητα του \tl{Jacobi}
    \begin{equation*}
        [X, [Y, Z]] + [Y, [Z, X]] + [Z, [X, Y]] = 0.
    \end{equation*}
\end{exercise}
\begin{solution}{3}
    Έστω δύο διανυσματικά πεδία $X,Y$ σε μία πολλαπλότητα $M^n$ και για μία συνάρτηση
    $f \in C^{\infty}(M^n)$ τότε η αγκύλη \tl{Lie} είναι
    \begin{equation*}
        [X, Y]f = YXf - XYf.
    \end{equation*}
    Σύμφωνα με το παραπάνω υπολογίζουμε τον πρώτο όρο της ταυτότητας του
    \tl{Jacobi} να είναι
    \begin{align}\label{eq:3_1}
        [X, [Y, Z]]f &= [Y, Z]Xf - X[Y, Z]f\nonumber \\
        &= (ZY - YZ)Xf - X(ZY - YZ)f \nonumber \\
        &= ZYXf - YZXf - XZYf + XYZf.
    \end{align}
    Αντίστοιχα ο δεύτερος όρος είναι
    \begin{align}\label{eq:3_2}
        [Y, [Z, X]]f &= [Z, X]Yf - Y[Z, X]f\nonumber \\
        &= (XZ - ZX)Yf - Y(XZ - ZX)f \nonumber \\
        &= XZYf - ZXYf - YXZf + YZXf,
    \end{align}
    και ο τελευταίος είναι
    \begin{align}\label{eq:3_3}
        [Z, [X, Y]]f &= [X, Y]Zf - Z[X, Y]f\nonumber \\
        &= (YX - XY)Zf - Z(YX - XY)f \nonumber \\
        &= YXZf - XYZf - ZYXf + ZXYf.
    \end{align}
    Αντικαθιστώντας τις σχέσεις \eqref{eq:3_1}, \eqref{eq:3_2} και
    \eqref{eq:3_3} στην ταυτότητα του \tl{Jacobi} παρατηρούμε ότι όλοι οι όροι
    διαγράφονται και επομένως η ταυτότητα ικανοποιείται.
\end{solution}
\begin{exercise}{4}
    Για $B \in SO(3, \mathbb{R})$ δείξτε ότι έχει μία ιδιοτιμή $\lambda = 1$.
\end{exercise}
\begin{solution}{4}
    Η ορθογώνια ομάδα ορίζεται
    \begin{equation*}
        O(n, \mathbb{R}) = \{B \in GL(n, \mathbb{R}) : B^{-1} = B^T \},
    \end{equation*}
    και η ειδική ορθογώνια ομάδα, που είναι μία υποομάδα της ορθογώνιας ομάδας ορίζεται
    \begin{equation*}
        SO(n, \mathbb{R}) = \{B \in O(n, \mathbb{R}) : \text{\tl{det}}(B) = 1 \}.
    \end{equation*}
    Είναι γνωστό ότι για οποιοδήποτε $n \times n$ μητρώο ισχύει ότι
    \begin{equation*}
        \text{\tl{det}}(B) = \lambda_1 \lambda_2 \cdots \lambda_n = 1,
    \end{equation*}
    όπου $\lambda_i$ είναι οι ιδιοτιμές του $B$. Επίσης το χαρακτηριστικό
    πολυώνυμο του μητρώου είναι
    \begin{equation}\label{eq:4_poly}
        p(\lambda) = \text{\tl{det}}\left[ \lambda I - B \right] = b_0 \lambda^n +
        b_1 \lambda_1^{n-1} + \dots + b_{n-1}\lambda + b_n.
    \end{equation}
    Οι συντελεστές $b_i$ της παραπάνω σχέσης είναι πραγματικοί αριθμοί και
    επομένως αν $z$ είναι μία μιγαδική λύση που ικανοποιεί τη \eqref{eq:4_poly}
    τότε και η συζυγής μιγαδική λύση $\bar{z}$ θα ικανοποιεί τη
    \eqref{eq:4_poly}. Αν $B \in O(n, \mathbb{R})$ τότε ο μετασχηματισμός που
    επιβάλει το μητρώο διατηρεί το εσωτερικό γινόμενο. Για τα διανύσματα
    $u, v$ το εσωτερικό γινόμενο είναι
    \begin{equation*}
        u \cdot v = u^Tv,
    \end{equation*}
    και η επίδραση του μητρώου $B$ είναι
    \begin{equation*}
        Bu \cdot Bv = (Bu)^T Bv = u^T B^T Bv = u^Tv.
    \end{equation*}
    Έτσι σύμφωνα με το παραπάνω μπορούμε να πούμε ότι το μητρώο $B$ δεν
    επηρεάζει το μέτρο ενός διανύσματος $u$ καθώς αυτό ορίζεται ως
    $u = \sqrt{u \cdot u}$. Επομένως αν $u$ είναι το
    ιδιοδιάνυσμα του μητρώου $B$ τότε ισχύει
    \begin{equation*}
        \left|B u \right| = \left|u \right|,
    \end{equation*}
    και άρα ισχύει
    \begin{equation*}
        \left|B u \right| = \left|\lambda u \right| = \left| u \right|,
    \end{equation*}
    που σημαίνει ότι $| \lambda | = 1$. Τελικά, σύμφωνα με αυτά που
    αναφέρθηκαν παραπάνω, οι πιθανές ρίζες του χαρακτηριστικού πολυωνύμου είναι
    $(1, 1, 1)$ ή $(1, -1, -1)$ ή $(1, z, \bar{z})$, και άρα σίγουρα για μία
    ιδιοτιμή ισχύει ότι $\lambda = 1$.
\end{solution}
\end{document}
