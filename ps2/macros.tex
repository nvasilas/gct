\newcommand{\tl}[1]{\textlatin{#1}}
\newcommand{\gr}[1]{\greektext{#1}}
\newcommand{\greek}{\selectlanguage{greek}}
\newcommand{\eng}{\selectlanguage{english}}

\usepackage{enumitem}
\newenvironment{alphenum}
{\begin{enumerate}
    [itemsep=-0.5ex, topsep=1.0ex, leftmargin=\parindent, align=left,
    labelwidth=!, label=\((\alph*)\)]}
{\end{enumerate}}

\newenvironment{romenum}
{\begin{enumerate}
    [itemsep=-5pt, leftmargin=*, labelsep=1em, align=left, labelwidth=!,label=\((\roman*)\)]}
{\end{enumerate}}

\usepackage{mathtools}
\newcommand\defeq{\coloneq}
\newcommand\eqdef{\eqcolon}

\newcommand\proofpt[1]{\({(#1)}\)}

\newcommand\inner[2]{\langle#1,\, #2\rangle}

\newcommand\dual[1]{#1^{*}}
\newcommand\algdual[1]{#1^{\#}}

\newcommand\lmaps{\mathcal{B}}

\newcommand\realf{R}
\newcommand\euclr[1]{\realf^{#1}}
\newcommand\deuclr[1]{\dual{(\euclr{#1})}}

\newcommand\restr[2]{#1|_#2}
\newcommand\compf{\mathbb{C}}
\newcommand\matrx[1]{\mathrm{#1}}
\newcommand\pcmap[1]{PC^{#1}}
\newcommand\cdiff[1]{C^{#1}}
\newcommand\metric{d}
\newcommand\di[2]{\metric(#1, #2)}

\newcommand{\vc}[1]{\bm{#1}} % this is a vector
\newcommand{\mt}[1]{\boldsymbol{#1}} % this is a matrix

\DeclareMathOperator\rank{rank}
\DeclareMathOperator\vspan{span}
\DeclareMathOperator\co{co}
\DeclareMathOperator\aff{aff}
\DeclareMathOperator\core{cor}
\DeclareMathOperator\icr{icr}
\DeclareMathOperator\diag{diag}
\DeclareMathOperator\minimize{minimize}
\DeclareMathOperator\argmin{arg\,min}
\DeclareMathOperator\der{D}
\DeclareMathOperator\qi{qi}
\DeclareMathOperator\qci{qci}
\DeclareMathOperator{\tr}{tr}

\newcommand\cseg[2]{[#1, #2]}
\newcommand\sseg[2]{[#1, #2)}
\newcommand\oseg[2]{(#1, #2)}

\newcommand\eps{\varepsilon}
\newcommand\dsum{\oplus}

\newcommand\norm[2][{}]{\|#2\|_{#1}}

\makeatletter
\newcommand\suchthat{%
    \@ifstar
    {\mathrel{}\middle|\mathrel{}}
    {\mid}%
}
\makeatother
\makeatletter
\newcommand\setbuild[2]{
    \@ifstar
    {\left\{#1\suchthat* #2\right\}}%
    {\{#1\suchthat #2\}}%
}
\makeatother
\newcommand\ball[2]{B_{#1}(#2)}
\newcommand\cball[2]{\closure{B}_{#1}(#2)}
\newcommand\oball[1]{B_{#1}}
\newcommand\ocball[1]{\bar{B}_{#1}}
\newcommand\compl[1]{#1^{\mathrm{c}}}
\newcommand\closure[2][.0]
{
    \mkern#1mu
    \overline{\mkern-#1mu#2}
}
\newcommand\interior[1]{{#1}^{\circ}}
\newcommand\boundary[1]{\partial#1}
\newcommand\voidset{\varnothing}

\newcommand\nulls[1]{\mathcal{N}(#1)}
\newcommand\range[1]{\mathcal{R}(#1)}

\newcommand\refthrm[2][]{Theorem \ifx\\#1\\\ref{#2}\else\ref{#2} \ref{#1}\fi}
\newcommand\refprop[2][]{Proposition \ifx\\#1\\\ref{#2}\else\ref{#2} \ref{#1}\fi}
\newcommand\reflemma[2][]{Lemma \ifx\\#1\\\ref{#2}\else\ref{#2} \ref{#1}\fi}

\renewcommand\theequation{\arabic{equation}}
\newtheoremstyle{thrm_style}
{} %spaceabove
{} %spacebelow
{\itshape} %bodyfont
{} %indent
{\bfseries} %headfont
{} %headpunctuation
{1em} %headspace
{\thmnumber{#2\kern1em}\thmname{#1}\thmnote{ (#3)}} %headspec
\theoremstyle{thrm_style}
\newtheorem{thrm}{\gr{Θεώρημα}}
\newtheorem{prop}[thrm]{\gr{Πρόταση}}
\newtheorem{cor}[thrm]{\gr{Πόρισμα}}
\newtheorem{lemma}[thrm]{\gr{Λήμμα}}

\newcommand\theoremname{}
\newtheorem{genericthrm}[thrm]{\theoremname}
\newenvironment{namedthrm}[1]
{\renewcommand\theoremname{#1}
\begin{genericthrm}}
{\end{genericthrm}}

\newtheoremstyle{defn_style}
{} %spaceabove
{} %spacebelow
{} %bodyfont
{} %indent
{\bfseries} %headfont
{} %headpunctuation
{1em} %headspace
{\thmnumber{#2\kern1em}\thmname{#1}\thmnote{ (#3)}} %headspec
\theoremstyle{defn_style}
\newtheorem{defn}[thrm]{\gr{Ορισμός}}
\newtheorem{exmp}[thrm]{\gr{Παράδειγμα}}

\newcommand\defnname{}
\newtheorem{genericdefn}[thrm]{\defnname}
\newenvironment{nameddefn}[1]
{\renewcommand\defnname{#1}
\begin{genericdefn}}
{\end{genericdefn}}

\newtheoremstyle{rem_style}
{} %spaceabove
{} %spacebelow
{} %bodyfont
{} %indent
{\bfseries} %headfont
{.} %headpunctuation
{.5em} %headspace
{\thmnumber{#2\kern1em}\thmname{#1}\thmnote{ (#3)}} %headspec
\theoremstyle{rem_style}
\newtheorem*{rem}{Remark}
\newtheorem*{note}{Note}

\renewcommand\qed{\unskip\nobreak\quad\qedsymbol}
\renewcommand\qedsymbol{\rule{1ex}{1.6ex}}

\newenvironment{exercise}[2][Άσκηση]
{\begin{trivlist}
    \item[\hskip \labelsep {\bfseries #1}\hskip \labelsep {\bfseries #2.}]}
{\end{trivlist}}

\newenvironment{solution}[2][Λύση]
{\begin{trivlist}
    \item[\hskip \labelsep {\bfseries #1}\hskip \labelsep {\bfseries #2.}]}
{\end{trivlist}}
