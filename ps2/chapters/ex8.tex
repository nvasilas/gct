\begin{exercise}{2016/17 8}
    Θεωρούμε το σύστημα ελέγχου στο επίπεδο, όπου είναι διαθέσιμα δύο σταθερά
    διανυσματικά πεδία \( X = \frac{\partial}{\partial x} +
    2\frac{\partial}{\partial y} \) και \( Y = 2\frac{\partial}{\partial x}
    + \frac{\partial}{\partial y} \) και επιτρέπεται να προχωρούμε με το ένα ή
    το άλλο διανυσματικό πεδίο, ή με τα αντίστροφά τους, δηλαδή μόνο με τις ροές
    \( \phi^X \) και \( \phi^Y \) των \( X \) και \( Y \).
    \begin{enumerate}[label= (\alph*)]
        \item Δείξτε ότι το \emph{προσβάσιμο σύνολο} \( \mathcal{R}(\vc{0}, T)
            \) των σημείων που μπορούμε να φτάσουμε σε χρόνο \( \leq T \) από το
            αρχικό σημείο \( \vc{0} \) είναι το \( \{x\in\mathbb{R}^2: \|Bx\|_1
            \leq c \} \), όπου \( \|x\|_1 = |x| + |y| \) είναι η \(1-\)νόρμα και
            \( B \) κατάλληλος \( 2 \times 2 \) πίνακας. Βρείτε τον \( B \) και
            τη σταθερά \( c \).
        \item Εάν τώρα περιορίσουμε το χώρο κατάστασης στο κλειστό τετράγωνο με
            κέντρο το \( \vc{0} \) και πλάτος \( 2 \) (δεν επιτρέπεται να βγούμε
            από αυτόν), δείξτε ότι κάθε σημείο του εσωτερικού του είναι
            προσβάσιμο, αλλά για κάποια σημεία στο σύνορο, δεν μπορούμε να
            φτάσουμε με πεπερασμένο αριθμό αλλαγών ροής. Ποια είναι τα σημεία
            αυτά;
    \end{enumerate}
\end{exercise}
\begin{solution}{2016/17 8}
    (α). Το διανυσματικό πεδίο \( X \) αντιστοιχεί στις διαφορικές εξισώσεις
    \[
        \dot{x} = 1, \quad \dot{y} = 2,
    \]
    που μας δίνει τις λύσεις
    \[
        x = t + x_0, \quad y = t + y_0.
    \]
    Αυτό σε μορφή ροής μπορεί να γραφτεί
    \[
        \phi^X_t(x_0, y_0) = (t + x_0, 2t + y_0),
    \]
    ή πιο γενικά χωρίς το δείκτη με το μηδέν
    \[
        \phi^X_t(x, y) = (t + x, 2t + y), \quad t \in \mathbb{R}.
    \]
    Αντίστοιχα για το διανυσματικό πεδίο \( Y \) προκύπτει η ροή
    \[
        \phi^Y_t(x, y) = (2t + x, t + y), \quad t \in \mathbb{R}.
    \]

    Παρατηρούμε ότι το να κινηθούμε πίσω στο χρόνο είναι το ίδιο με το να
    κινηθούμε μπροστά στο χρόνο με το αντίθετο διανυσματικό πεδίο. Επίσης,
    επειδή τα πεδία είναι γραμμικά και σταθερά η σειρά με την οποία θα κινηθούμε
    δεν παίζει ρόλο. Αυτό που παίζει ρόλο είναι για πόσο χρόνο θα κινηθούμε με το
    κάθε πεδίο. Για παράδειγμα, το να κινηθούμε με την ροή του \( Y \) για χρόνο
    \( c \), μετά με την ροή του \( X \) για χρόνο \( b \) και τέλος με την
    ροή του \( Y \) για χρόνο \( a \), είναι το ίδιο με το να κινηθούμε με την
    ροή του \( X \) για χρόνο \( b \), μετά με την ροή του \( Y \) για χρόνο
    \( c \) και τέλος με την ροή του \( Y \) για χρόνο \( a \). Δηλαδή
    ισχύει
    \[
        \phi^Y_{c} \circ \phi^X_{b} \circ \phi^Y_{a} =
        \phi^X_{b} \circ \phi^Y_{c} \circ \phi^Y_{a}.
    \]

    Ορίζουμε τα χρονικά διαστήματα που κινηθήκαμε συνολικά με την κάθε ροή.
    Έτσι συμβολίζουμε με \( t_x \in \mathbb{R} \) το συνολικό χρόνο που
    κινηθήκαμε με τη ροή του \( X \), δηλαδή την \( \phi^{X}_{t_x} \).
    Αντίστοιχα, με \( t_y \in \mathbb{R} \) το συνολικό χρόνο που κινηθήκαμε
    με τη ροή του \( Y \), δηλαδή την \( \phi^{Y}_{t_y} \).

    Το προσβάσιμο σύνολο είναι η σύνθεση όλων των δυνατών συνδυασμών των ροών.
    Όμως διότι στην περίπτωση αυτή, δεν παίζει ρόλο ο τρόπος με των οποίο θα
    κινηθούμε αλλά μόνο ο συνολικός χρόνος παραμονής στην κάθε ροή, έτσι έχουμε
    ότι το προσβάσιμο σύνολο είναι
    \begin{align}\label{eq:ex8_flow}
        \phi^X_{t_x} \circ \phi^Y_{t_y} (x, y) &=
        \phi^X_{t_x} (2t_y + x, t_x + y) \\
        &= (t_x + 2t_y + x, 2t_x + t_y + y).
    \end{align}
    Η παραπάνω σχέση δέχεται σαν είσοδο κάποια \( (x, y) \in \mathbb{R}^2 \) και
    δίνει σαν έξοδο το παραπάνω διάνυσμα. Οπότε είναι ισοδύναμη σχέση με τη
    σύνολο της εκφώνησης. Αν πάρουμε την \(1-\)νόρμα για το αρχικό σημείο
    \( (0, 0) \) προκύπτει
    \[
        |t_x + 2t_y| + |2t_x + t_y| \leq |t_x| + 2|t_y|  + 2|t_x| + |t_y|,
    \]
    όπου το δεύτερο μέλος της ανισότητας ισούται
    \[
        3|t_x| + 3|t_y| = 3(|t_x| + |t_y|) = 3T,
    \]
    όπου \( T \) είναι η συνολική χρονική διάρκεια που κινηθήκαμε με τα διανυσματικά πεδία.

    (β). Αν τώρα περιορίσουμε το χώρο κατάστασης στον κλειστό τετράγωνο με
    κέντρο το \( 0 \) και πλάτος το \( 2 \) τότε για να φτάσουμε στο σημείο,
    του χώρου κατάστασης, με συντεταγμένες \( (a, b) \) από τη
    σχέση~\eqref{eq:ex8_flow} ισχύει
    \begin{align*}
        t_x + 2t_y &= a \\
        2t_x + t_y &= b.
    \end{align*}
    Λύνοντας τις παραπάνω ως προς \( t_x, t_y \) προκύπτει
    \begin{equation}\label{eq:ex8_tx_ty}
        t_x = \frac{2b - a}{3}, \quad t_y = \frac{2a - b}{3}.
    \end{equation}

    Είναι προφανές ότι εφόσον περιορίσαμε το χώρο στο κλειστό τετράγωνο, ο
    μέγιστος χρόνος που μπορούμε να κινηθούμε με το κάθε πεδίο, είτε μπροστά
    είτε πίσω, είναι περιορισμένος. Δηλαδή ισχύουν τα όρια
    \begin{equation}\label{eq:ex8_tx_tx_limits}
        -1 \leq t_x \leq 1, \quad -1 \leq t_y \leq 1.
    \end{equation}
    Ακόμη, εφόσον μπορούμε να κινηθούμε εντός του τετραγώνου ή στο όριο του τα
    σημεία \( (a, b) \) είναι εντός των ορίων
    \[
        -1 \leq a \leq 1, \quad -1 \leq b \leq 1.
    \]

    Για παράδειγμα, μόνο με το διανυσματικό πεδίο \( X \) φτάνω τα σημεία όταν
    \( t_y = 0 \), ή όταν \( 2a = b \) δηλαδή για όλα τα σημεία που βρίσκονται
    στην ευθεία \( y = 2x \) που διαγράφεται στο τετράγωνο. Έτσι αν θέλω να πάω
    στο σημείο \( (0.5, 1) \), αυτό επιτυγχάνεται μονάχα με τη ροή του \( X \).

    Θα δείξουμε ότι κάθε σημείο του εσωτερικού είναι προσβάσιμο θεωρώντας ότι
    υπάρχει σημείο \( (a, b) \) που ανήκει στο χώρο κατάστασης αλλά δεν είναι
    προσβάσιμο. Αυτό σημαίνει ότι οι σχέσεις~\eqref{eq:ex8_tx_ty} θα δώσουν λύση
    εκτός των ορίων~\eqref{eq:ex8_tx_tx_limits}. Επομένως για το \( t_x \) θα
    πρέπει να ισχύει
    \[
        t_x < -1, \quad \text{\gr{ή}} \quad t_x > 1.
    \]
    Αντικαθιστώντας προκύπτει για την ανίσωση στα αριστερά
    \begin{align*}
        \frac{2b - a}{3} &< -1 \\
        2b - a &< -3,
    \end{align*}
    όπου οι ακραίες τιμές που μπορεί να πάρει η παραπάνω, δηλαδή η ελάχιστη
    τιμή, διότι τα \( (a, b) \) είναι εντός του τετραγώνου ή στο σύνορο, είναι
    \( a = 1, b = -1 \) και έτσι καταλήγουμε ότι \( -3 < -3 \), άτοπο.

    Με την ίδια λογική, αντικαθιστώντας για την ανίσωση στα δεξιά έχουμε
    \begin{align*}
        \frac{2b - a}{3} &> 1 \\
        2b - a &> 3,
    \end{align*}
    όπου με την ίδια αιτιολόγηση, οι ακραίες τιμές που μπορεί να πάρει η παραπάνω,
    είναι \( a = -1, b = 1 \) και έτσι καταλήγουμε ότι \( 3 > 3 \), άτοπο.

    Αντίστοιχα για το \( t_y \) θα πρέπει να ισχύει
    \[
        t_y < -1, \quad \text{\gr{ή}} \quad t_y > 1.
    \]
    Αντικαθιστώντας προκύπτει για την ανίσωση στα αριστερά
    \begin{align*}
        \frac{2a - b}{3} &< -1 \\
        2a - b &< -3,
    \end{align*}
    όπου οι ακραίες τιμές που μπορεί να πάρει η παραπάνω, δηλαδή η ελάχιστη
    τιμή, διότι τα \( (a, b) \) είναι εντός του τετραγώνου ή στο σύνορο, είναι
    \( a = -1, b = 1 \) και έτσι καταλήγουμε ότι \( -3 < -3 \), άτοπο.

    Με την ίδια λογική, αντικαθιστώντας για την ανίσωση στα δεξιά έχουμε
    \begin{align*}
        \frac{2a - b}{3} &> 1 \\
        2a - b &> 3,
    \end{align*}
    όπου με την ίδια αιτιολόγηση, οι ακραίες τιμές που μπορεί να πάρει η παραπάνω,
    είναι \( a = 1, b = -1 \) και έτσι καταλήγουμε ότι \( 3 > 3 \), άτοπο.

    Άρα καταλήγουμε ότι κάθε σημείο του τετραγώνου είναι προσβάσιμο.

    Παρόλο που κάθε σημείο είναι προσβάσιμο, φτάνουμε σε κάποια σημεία
    \enquote*{πιο δύσκολα} σε σχέση με άλλα. Έτσι όπως είναι ορισμένα τα
    διανυσματικά πεδία \( X \) και \( Y \), επειδή μπορούμε να κινηθούμε μόνο με
    τις ροές τους, βρισκόμενοι σε ένα σημείο μπορούμε να κινηθούμε στο επάνω
    δεξιά και κάτω αριστερά τεταρτημόριο, που ορίζει το εκάστοτε σημείο. Ακόμη,
    τα σημεία που δεν μπορούμε να φτάσουμε με πεπερασμένο αριθμό αλλαγών ροής
    είναι τα σημεία όπου \( t_x \) και \( t_y \) παίρνουν ακραίες τιμές.

    Άρα για τα ακραία σημεία \( t_x = 1, t_y = 1 \) προκύπτει
    \( t_x = (2b - a)/3 = 1 \) και \( t_y = (2a - b)/3 = 1 \), που σημαίνει ότι
    \( a = 3, b = 3 \), άτοπο. Αντίστοιχα, για \( t_x = -1, t_y = -1 \) προκύπτει
    \( t_x = (2b - a)/3 = -1 \) και \( t_y = (2a - b)/3 = -1 \), που σημαίνει ότι
    \( a = -3, b = -3 \), άτοπο.

    Συνεχίζοντας, για \( t_x = -1, t_y = 1 \) προκύπτει \( t_x = (2b - a)/3 = -1 \) και
    \( t_y = (2a - b)/3 = 1 \), που σημαίνει ότι \( a = 1, b = -1 \), που είναι
    ένα σημείο που δεν μπορούμε να φτάσουμε με πεπερασμένο αριθμό αλλαγών ροής.
    Τέλος, για \( t_x = 1, t_y = -1 \) προκύπτει \( t_x = (2b - a)/3 = 1 \) και
    \( t_y = (2a - b)/3 = -1 \), που σημαίνει ότι \( a = -1, b = 1 \), που είναι
    επίσης ένα σημείο που δεν μπορούμε να φτάσουμε με πεπερασμένο αριθμό αλλαγών
    ροής. Τα σημεία αυτά επιβεβαιώνουν  την περιγραφή για τα
    \enquote*{δύσκολα} σημεία που κάναμε παραπάνω.
\end{solution}
