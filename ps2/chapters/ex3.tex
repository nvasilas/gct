\begin{exercise}{2016/17 3}
    \emph{Οι εξισώσεις \tl{Lorentz}}: Όπως είναι γνωστό το παρακάτω απλό σύστημα
    σε τρεις διαστάσεις που μελετήθηκε το \( 1963 \) από τον \tl{E. Lorenz}
    παρουσιάζει \enquote*{χαοτική συμπεριφορά} για κάποιες τιμές των παραμέτρων
    του.
    \begin{align*}
        \dot{x} &= \sigma(y - x) \\
        \dot{y} &= \rho x - y - xz \\
        \dot{z} &= \beta z + xy, \quad ( \sigma, \rho, beta > 0).
    \end{align*}
    Προς το παρόν, θα δούμε κάποια βασικά χαρακτηριστικά του.
    \begin{enumerate}[label= (\alph*)]
        \item Βρείτε τα σημεία ισορροπίας του και την εξάρτησή τους από τις
            παραμέτρους.
        \item Δείξτε ότι ο κάθετος άξονας \( z \) είναι αναλλοίωτη ευθεία για το
            σύστημα και περιγράψτε τη δυναμική πάνω του. Επίσης, δείξτε ότι το
            σύστημα έχει τη συμμετρία \( (x, y, z) \to (-x, -y, z) \).
        \item Δείξτε ότι
            \begin{enumerate}[label= (\roman*)]
                \item για \( 0 < \rho < 1 \), το σημείο ισορροπίας στο \( 0 \)
                    είναι ολικά ασυμπτωτικά ευσταθές, κάνοντας χρήση της
                    συνάρτησης \tl{Lyapunov}
                    \begin{equation*}
                        V(x, y, z) = \rho x^2 + \sigma(y^2 + z^2).
                    \end{equation*}
                \item για \( \rho > 1 \), το σημείο ισορροπίας \( 0 \) έχει
                    μονοδιάστατη ασταθή πολλαπλότητα \( W^u(0) \).
            \end{enumerate}
    \end{enumerate}
\end{exercise}
\begin{solution}{2016/17 3}
    (α). Τα σημεία ισορροπίας υπολογίζονται από τις σχέσεις
    \begin{align*}
        0 &= \sigma(y - x) \\
        0 &= \rho x - y - xz \\
        0 &= \beta z + xy
    \end{align*}
    Από την πρώτη σχέση προκύπτει
    \[
        y = x.
    \]
    Με αντικατάσταση αυτής στην τρίτη
    και λύνοντας ως προς \( z \) προκύπτει
    \[
        z = \frac{x^2}{\beta}.
    \]
    Τέλος, με αντικατάσταση των δύο παραπάνω σχέσεων στην δεύτερη έχουμε
    \[
        0 = \rho x - x \frac{x^2}{\beta} = x \left( \rho -1 - \frac{x^2}{\beta}
        \right),
    \]
    που σημαίνει \( x = 0 \), ή
    \begin{align*}
        0 &= \rho -1 - \frac{x^2}{\beta} \\
        x^2 &= \beta(\rho - 1),
    \end{align*}
    και για \( \rho > 1 \) προκύπτει \( x = \pm\sqrt{\beta(\rho - 1)} \).

    Τελικά, τα σημεία ισορροπίας είναι το \( \left(0, 0, 0\right) \) και αν \( \rho > 1 \) έχουμε
    και τα σημεία \( \left(\sqrt{\beta(\rho - 1)}, \sqrt{\beta(\rho - 1)}, \rho
    -1\right) \) και \( \left( -\sqrt{\beta(\rho - 1)}, -\sqrt{\beta(\rho - 1)},
    \rho -1 \right) \).

    Για να διαπιστώσουμε τη μορφή των σημείων ισορροπίας θα πάρουμε τη
    γραμμικοποίηση του συστήματος. Αν γράψουμε το σύστημα στη μορφή
    \begin{equation*}
        \begin{pmatrix}
            \dot{x} \\
            \dot{y} \\
            \dot{z}
        \end{pmatrix} = F(x, y, z) =
        \begin{pmatrix}
            \sigma(y - x) \\
            \rho x - y - xz \\
            \beta z + xy
        \end{pmatrix},
    \end{equation*}
    τότε η Ιακωβιανή είναι
    \begin{equation*}
        DF(x, y, z) =
        \begin{pmatrix}
            -\sigma & \sigma & 0 \\
            \rho - z & -1 & -x \\
            y & x & -\beta \\
        \end{pmatrix}.
    \end{equation*}
    Έτσι για την αρχή ισχύει
    \begin{equation*}
        DF(0, 0, 0) =
        \begin{pmatrix}
            -\sigma & \sigma & 0 \\
            \rho & -1 & 0 \\
            0 & 0 & -\beta \\
        \end{pmatrix}.
    \end{equation*}
    Είναι εμφανές ότι μία ιδιοτιμή είναι \( \lambda_1 = -\beta < 0\). Για να
    βρούμε τις άλλες δύο θα πάρουμε την ορίζουσα
    \begin{equation*}
        \det
        \begin{pmatrix}
            -\sigma - \lambda & \sigma \\
            \rho & -1 - \lambda \\
        \end{pmatrix}=
        (-\sigma - \lambda)(-1 - \lambda) - \sigma \rho,
    \end{equation*}
    που τελικά δίνει
    \begin{equation*}
        \lambda^2 + \lambda(\sigma + 1) + \sigma(1 - \rho) = 0.
    \end{equation*}
    Η διακρίνουσα της παραπάνω είναι
    \begin{equation*}
        \Delta = {(\sigma + 1)}^2 - 4\sigma(1 - \rho),
    \end{equation*}
    και οι ιδιοτιμές είναι
    \begin{equation*}
        \lambda_{2,3} = \frac{-(\sigma + 1) \pm \sqrt{\Delta}}{2}.
    \end{equation*}
    Αν \( \lambda_2 \) είναι η ιδιοτιμή που προκύπτει από το θετικό
    πρόσημο που είναι μπροστά από την τετραγωνική ρίζα και αντίστοιχα
    \( \lambda_3 \) η άλλη ιδιοτιμή, τότε προφανώς  \( \lambda_3 < 0 \). Θα
    δούμε, για την ιδιοτιμή \( \lambda_2 \). Οι ρίζες της είναι
    \begin{equation*}
        -(\sigma + 1) + \sqrt{{(\sigma + 1)}^2 - 4\sigma(1 - \rho)} = 0,
    \end{equation*}
    όπου υψώνοντας στο τετράγωνο προκύπτει
    \begin{equation*}
        {(\sigma + 1)}^2 - 4\sigma(1 - \rho) = {(\sigma + 1)}^2,
    \end{equation*}
    ή
    \begin{equation*}
        - 4\sigma(1 - \rho) = 0,
    \end{equation*}
    και επειδή \( \sigma > 0 \), προκύπτει \( \rho = 1 \).

    Άρα \( \lambda_2 < 0 \), όταν \( 0 < \rho < 1 \) και τότε, διότι
    \( \lambda_1, \lambda_3 < 0 \), από το θεώρημα
    \tl{Hartman-Grobman} το \( (0, 0, 0) \) είναι ασυμπτωτικά ευσταθές.

    Όταν \( \rho > 1 \), τότε \( \lambda_2 > 0 \) και επειδή
    \( \lambda_1, \lambda_3 < 0 \), η αρχή από το θεώρημα
    \tl{Hartman-Grobman} είναι σάγμα. Αυτό αποδεικνύει το ερώτημα (γ)
    \tl{ii}.

    Όταν \( \rho = 1 \), τότε \( \lambda_2 = 0 \) και \( \lambda_3 = -(\sigma +
    1) \) αλλά επειδή έχουμε μη υπερβολικό σημείο ισορροπίας δεν μπορούμε να
    χρησιμοποιήσουμε το θεώρημα \tl{Hartman-Grobman}. Η αντίστοιχη ανάλυση για
    τα άλλα δύο σημεία ισορροπίας είναι ιδιαίτερα πολύπλοκη και θα παραλειφθεί.

    (β). Το σύστημα έχει τη συμμετρία \( (x, y, z) \to (-x, -y, z) \) και
    φαίνεται εύκολα αν εφαρμόσουμε το μετασχηματισμό. Έτσι
    \begin{align*}
        -\dot{x} &= \sigma(-y + x) \\
        -\dot{y} &= -\rho x + y + xz \\
        \dot{z} &= -\beta z + (-x)(-y),
    \end{align*}
    όπου όμως το παραπάνω ισούται με το αρχικό
    \begin{align*}
        \dot{x} &= \sigma(y - x) \\
        \dot{y} &= \rho x - y - xz \\
        \dot{z} &= -\beta z + xy,
    \end{align*}
    και αυτό επιβεβαιώνει τη συμμετρία ως προς τον άξονα \( z \).

    Ένα σύνολο \( K \subset \mathbb{R}^n \) είναι αναλλοίωτο για τη ροή ενός
    διανυσματικού πεδίου εάν \( x \in K \Rightarrow \phi(x, t) \in K \).
    Επομένως, για να ένα σημείο \( z \in (0, 0, z) \) έχουμε
    \begin{align*}
        \dot{x} &= 0 \\
        \dot{y} &= 0 \\
        \dot{z} &= -\beta z,
    \end{align*}
    και άρα η ροή του διανυσματικού πεδίου, για οποιοδήποτε σημείο πάνω στον
    άξονα \( z \), παραμένει στον άξονα \( z \) για κάθε \( t \) και άρα είναι
    αναλλοίωτη ευθεία για το σύστημα.

    (γ) \tl{i}. Θα χρησιμοποιήσουμε τη συνάρτηση \tl{Lyapunov}
    \begin{equation*}
        V(x, y, z) = \rho x^2 + \sigma(y^2 + z^2),
    \end{equation*}
    για \( 0 < \rho < 1 \) και για το σημείο \( (0, 0, 0) \). Αρχικά στο σημείο
    αυτό προφανώς ισχύει \( V(0, 0, 0) = 0 \) και για \( (x, y, z) \neq (0, 0,
    0) \) βλέπουμε ότι \( V(x, y, z) > 0 \). Στη συνέχεια θα υπολογίσουμε το
    ρυθμό μεταβολής της συνάρτησης \tl{Lyapunov}. Έτσι έχουμε
    \begin{align*}
        \frac{d}{dt}V(x(t), y(t), z(t))
        &= 2\rho x \dot{x} + 2\sigma y \dot{y} + 2 \sigma z \dot{z} \\
        &= 2\rho x (\sigma(y - x)) + 2\sigma y (\rho x - y - xz) + 2 \sigma z
        (-\beta z + xy) \\
        &= 2\rho x\sigma y - 2\rho\sigma x^2 + 2\sigma y\rho x - 2\sigma y^2 -
        2\sigma yxz - 2 \sigma\beta z^2 + 2\sigma zxy \\
        &= -2\sigma \left( -2\rho xy + \rho x^2 + y^2 + \beta z^2 \right) \\
        &= -2\sigma \left( -2\rho xy + \rho x^2 + y^2 + \beta z^2 + \rho^2 x^2 -
        \rho^2x^2\right) \\
        &= -2\sigma \left[ {(\rho x - y)}^2 + \beta z^2 + \rho x^2(1 -
        \rho)\right].
    \end{align*}
    Δεδομένου ότι \( 0 < \rho < 1 \), ισχύει \( \dot{V} < 0 \) καθώς όλοι οι
    όροι είναι τετραγωνικοί. Έτσι για κάθε \( (x, y, z) \neq (0, 0, 0) \) και
    συνεπώς είναι ολικά ασυμπτωτικά ευσταθές σημείο ισορροπίας.
\end{solution}
