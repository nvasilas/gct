\begin{exercise}{2016/17 6}
    \emph{Χαμιλτονιανοί πίνακες}: Ο πίνακας γραμμικοποίησης σε σημείο ισορροπίας
    Χαμιλτονιανού συστήματος στο επίπεδο,
    \begin{equation*}
        \dot{x} = \frac{\partial H}{\partial y}, \quad
        \dot{y} = -\frac{\partial H}{\partial x},
    \end{equation*}
    είναι
    \begin{equation*}
        \begin{pmatrix}
            \frac{\partial ^2 H}{\partial x\partial y} &
            \frac{\partial ^2 H}{\partial y^2} \\
            -\frac{\partial ^2 H}{\partial x^2} &
            -\frac{\partial ^2 H}{\partial x\partial y}
        \end{pmatrix},
    \end{equation*}
    και επομένως έχει μηδενικό ίχνος.

    \begin{enumerate}[label= (\alph*)]
        \item Δείξτε ότι η γραμμικοποίηση αυτή δίνει πάντα ή κέντρο ή σάγμα (με
            ίσες σε απόλυτη τιμές ιδιοτιμές).
        \item Ένας τετραγωνικός πίνακας \( A \) διάστασης \( m = 2n \) λέγεται
            \emph{Χαμιλτονιανός} εάν ικανοποιεί την ταυτότητα
            \[
                A^{T}J + JA = 0,
            \]
            όπου \( J \) είναι ο πίνακας
            \begin{equation*}
                \begin{pmatrix}
                    0 &
                    I_n \\
                    -I_n &
                    0
                \end{pmatrix}.
            \end{equation*}
            Δείξτε ότι κάθε τέτοιος \( A \) έχει μηδενικό ίχνος.
        \item Δείξτε ότι εάν \( \lambda \) είναι ιδιοτιμή Χαμιλτονιανού πίνακα
            \( A \), τότε και \( -\lambda \) είναι ιδιοτιμή. (Υπόδειξη: χρήση
                της ορίζουσας της ταυτότητας και θεώρηση του χαρακτηριστικού
            πολυωνύμου.)
        \item Βρείτε τη γενική μορφή Χαμιλτονιανού πίνακα σε δύο διαστάσεις,
            \( m = 2 \), και δείξτε ότι οι ιδιοτιμές του είναι ή \( \lambda =
            \pm i\omega \) ή \( \lambda_2 = -\lambda_1 \).

            Θεωρούμε τώρα μικρές διαταραχές του \( A \) εντός του συνόλου των
            Χαμιλτονιανών πινάκων. Δείξτε ότι εφόσον οι ιδιοτιμές είναι
            μη-μηδενικές, το διαταραγμένο σύστημα έχει σημεία ισορροπίας του
            ίδιου τύπου με το αρχικό (κέντρο ή συμμετρικό σάγμα).
    \end{enumerate}
\end{exercise}
\begin{solution}
    (α). Είναι γνωστό ότι χαρακτηριστικό πολυώνυμο στο επίπεδο δίνεται από τη
    σχέση
    \[
        p(\lambda) = \lambda^2 - \lambda\tr{A}+ \det{A},
    \]
    όμως επειδή \( \tr{A} = 0 \) προκύπτει
    \[
        p(\lambda) = \lambda^2 + \det{A}.
    \]
    Επομένως η διακρίνουσα είναι
    \[
        \Delta = -4\det{A}
    \]
    και η λύση
    \begin{equation}\label{eq:ex6_lambda}
        \lambda = \frac{\pm \sqrt{\Delta}}{2}.
    \end{equation}
    Έτσι παίρνουμε περιπτώσεις ανάλογα με την ορίζουσα του \( A \).

    Αν \( \det{A} < 0 \), τότε \( \Delta > 0 \) και άρα οι ιδιοτιμές θα είναι
    πραγματικές με τη μία θετική και την άλλη αρνητική, και φυσικά ίσες σε
    απόλυτη τιμή, όπως φαίνεται στη~\eqref{eq:ex6_lambda}.

    Αν \( \det{A} > 0 \), τότε \( \Delta < 0 \). Όμως επειδή το ίχνος είναι
    μηδέν, δηλαδή το πραγματικό μέρος των ιδιοτιμών είναι μηδέν, αυτό σημαίνει
    ότι θα έχουμε φανταστικές ιδιοτιμές και αυτό συνεπάγεται κέντρο.

    Η περίπτωση όπου \( \det{A} = 0\) δεν έχει νόημα, καθώς τότε μιλάμε για
    μηδενικό πίνακα.

    (β). Αν \( A \) πίνακας διάστασης \( m = 2n \), τότε μπορώ να τον γράψω στη
    μορφή
    \[
        A =
        \begin{pmatrix}
            a & b \\
            c & d
        \end{pmatrix},
    \]
    όπου κάθε μπλοκ \( a, b, c, d \) να είναι ένας \( n \times n \) πίνακας.
    Τότε με αντικατάσταση στην ταυτότητα και αναπτύσσοντας τις πράξεις έχουμε
    \begin{align*}
        A^{T}J + JA &=
        \begin{pmatrix}
            a & c \\
            b & d
        \end{pmatrix}
        \begin{pmatrix}
            0 & I_n \\
            -I_n & 0
        \end{pmatrix} +
        \begin{pmatrix}
            0 & I_n \\
            -I_n & 0
        \end{pmatrix}
        \begin{pmatrix}
            a & b \\
            c & d
        \end{pmatrix} \\
        &=\begin{pmatrix}
            -cI_n & aI_n \\
            -dI_n & bI_n
        \end{pmatrix} +
        \begin{pmatrix}
            I_{n}c & I_{n}d \\
            -I_{n}a & -I_{n}b
        \end{pmatrix}\\
        &=\begin{pmatrix}
            -cI_n + I_{n}c & aI_n + I_{n}d\\
            -dI_n - I_{n}a & bI_n -I_{n}b
        \end{pmatrix} \\
        &=\begin{pmatrix}
            0 & aI_n + I_{n}d\\
            -dI_n - I_{n}a & 0
        \end{pmatrix}.
    \end{align*}
    Όμως για να ισχύει η ταυτότητα θα πρέπει να ισχύει για τους πίνακες
    \( a \) και \( d \)
    \[
        a + d = 0,
    \]
    και άρα ο πίνακας \( A \) θα έχει μηδενικό ίχνος. Έτσι ο πίνακας \( A \)
    μπορεί να γραφτεί στη μορφή
    \[
        A =
        \begin{pmatrix}
            a & b \\
            c & -a
        \end{pmatrix}.
    \]

    (γ). Αρχικά παρατηρούμε ότι
    \[
        -J = \begin{pmatrix}
            0 & -I_n \\
            I_n & 0
        \end{pmatrix} = J^T,
    \]
    που σημαίνει ότι
    \[
        J^{T}J = (-J) (-J^T) = JJ^T =
        \begin{pmatrix}
            0 & I_n \\
            -I_n & 0
        \end{pmatrix}
        \begin{pmatrix}
            0 & -I_n \\
            I_n & 0
        \end{pmatrix} = I_m.
    \]
    Το παραπάνω σημαίνει πως \( J^T = J^{-1} \), δηλαδή ο \( J \) είναι ορθογώνιος
    πίνακας και συνεπώς οι ιδιοτιμές του είναι \( \pm 1 \). Πιο συγκεκριμένα, οι
    ιδιοτιμές του \( J \) είναι
    \[
        \det{J} = \det{\begin{pmatrix}0 & I_n\\ -I_n & 0\end{pmatrix}} =
        \det{\left( 0 - (-I_n)I_n \right)} = \det{I_n} = 1.
    \]

    Από την ταυτότητα των Χαμιλτονιανών πινάκων έχουμε
    \begin{align*}
        0 &= A^{T}J + JA \\
        -JA &= A^{T}J \\
        -A &= J^{-1}A^{T}J \\
        -A &= -JA^{T}J \\
        A &= JA^{T}J.
    \end{align*}

    Τώρα αν πάρουμε το χαρακτηριστικό πολυώνυμο του \( A \) προκύπτει
    \begin{align*}
        p(\lambda) &= \det{\left( A - \lambda I_m \right)}
        = \det{\left( JA^{T}J - \lambda J^{T}J \right)} \\
        &= \det{\left( JA^{T}J + \lambda J^{2} \right)}
        = \det{\left( J\left(A^{T} + \lambda I_m\right)J \right)} \\
        &= \det{\left( J \right)}
        \det{\left( A^T + \lambda I_m \right)}
        \det{\left( J \right)}\\
        &= \det{\left( A^T + \lambda I_m \right)}
        = \det{\left( A + \lambda I_m \right)^{T}}\\
        &= \det{\left( A + \lambda I_m \right)}
        = \det{\left( A - (-\lambda) I_m \right)}\\
        &= p(-\lambda).
    \end{align*}

    (δ). Άμεσα από τα παραπάνω έχουμε ότι αν \( \lambda \) είναι η μία ιδιοτιμή
    του \( A \) τότε η \( -\lambda \) θα είναι η άλλη ιδιοτιμή του. Αν \( A \)
    είναι δισδιάστατος πίνακας της μορφής
    \[
        A =
        \begin{pmatrix}
            a & b \\
            c & d
        \end{pmatrix},
    \]
    τότε θα ισχύει
    \begin{align*}
        \det{\begin{pmatrix}
                a - \lambda & b \\
                c & d - \lambda
        \end{pmatrix}} &=
        \det{\begin{pmatrix}
                a + \lambda & b \\
                c & d + \lambda
        \end{pmatrix}} \\
        (a - \lambda)(d - \lambda) - bc &=
        (a + \lambda)(d + \lambda) - bc \\
        ad - a\lambda - \lambda d + \lambda^2 &=
        ad + a\lambda + \lambda d + \lambda^2,
    \end{align*}
    που συνεπάγεται
    \[
        \lambda(a + d) = 0,
    \]
    και άρα \( \lambda = 0 \) ή στη γενικά \( d = -a \). Επομένως ο \( A \) έχει
    μηδενικό ίχνος και είναι της γενικής μορφής
    \[
        A =
        \begin{pmatrix}
            a & b \\
            c & -a
        \end{pmatrix}.
    \]
    Το χαρακτηριστικό πολυώνυμο του \( A \) είναι
    \[
        p(\lambda) = \lambda^2 + \det{A},
    \]
    με διακρίνουσα \( \Delta = -4 \det{A} \). Σύμφωνα με την ανάλυση που κάναμε
    στο ερώτημα (α), καταλήγουμε ότι οι ιδιοτιμές του πίνακα \( A \) είναι \(
    \lambda = \pm i\omega \) ή \( \lambda_2 = - \lambda_1 \).

    Σύμφωνα με τα παραπάνω, οι ιδιοτιμές του \( A \) θα είναι διακριτές.
    Συνεπώς, αν πάρουμε τη μορφή \tl{Jordan} του \( A \) στο χώρο \(
    M_n\left(\mathbb{C}\right) \) τότε θα οι ιδιοτιμές θα είναι τα διαγώνια
    στοιχεία. Αν τώρα διαταράξουμε \enquote*{λίγο} τα στοιχεία αυτά, τότε οι
    συντελεστές του χαρακτηριστικού πολυωνύμου του \( A \) θα διαταραχθούν
    \enquote*{λίγο}. Έτσι οι ρίζες του πολυωνύμου θα διαταραχθούν
    \enquote*{λίγο}. Συνεπώς, εφόσον ο πίνακας \( A \) έχει διακριτές ιδιοτιμές,
    πίνακες αυθαίρετα κοντά σε αυτόν θα έχουν τις ίδιες ιδιότητες με τον
    \( A \), και άρα τα σημεία ισορροπίας θα είναι του ίδιου τύπου με το αρχικό.
\end{solution}
