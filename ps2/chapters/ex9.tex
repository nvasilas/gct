\begin{exercise}{2016/17 9}
    Εξετάστε το ακόλουθο πρόβλημα: δίνεται γραμμικό υποψήφιο σύστημα
    \( \dot{x} = Ax \) στο \( \mathbb{R}^n \), με το \( 0 \) ασυμπτωτικά
    ευσταθές. Η απλούστερη μορφή υποψήφιας συνάρτησης \tl{Lyapunov} για το \( 0
    \) είναι
    \[
        V(x) = \frac{1}{2}x^{T}Qx,
    \]
    με \( Q = Q^T > 0 \) (θετικά ορισμένο, συμμετρικό). Γιατί;

    Δείξτε ότι πάντα υπάρχει τέτοιος \( Q \) και μάλιστα με \( \frac{dV}{dt} =
    -x^{T}Px \), με \( P \) επίσης θετικά ορισμένο συμμετρικό πίνακα.

    Δώστε ένα (μη-διαγώνιο!) παράδειγμα στο \( \mathbb{R}^3 \).
\end{exercise}
\begin{solution}{2016/17 9}
    Για να είναι η \( V \) συνάρτηση \tl{Lyapunov} στο \( 0 \) θα πρέπει να
    ικανοποιούνται τα παρακάτω. Πρώτον, προφανώς στο \( 0 \) ισχύει
    \[
        V(0) = 0.
    \]
    Δεύτερον, για κάθε \( x \neq 0 \) ισχύει ότι
    \[
        V(x) = \frac{1}{2}x^{T}Qx > 0,
    \]
    διότι \( Q \) είναι θετικά ορισμένος και συμμετρικός. Τρίτον, για \( x \neq
    0 \) ισχύει ότι
    \[
        \frac{dV}{dt} = \frac{1}{2}
        \frac{\partial \left( x^{T}Qx \right)}{\partial x}
        \frac{dx}{dt} =
        x^{T}QAx.
    \]
    Όμως γνωρίζουμε πως το \( 0 \) είναι ασυμπτωτικά ευσταθές. Ακόμα, επειδή το
    σύστημα είναι γραμμικό, αυτό συνεπάγεται πως ο πίνακας \( A \) θα είναι
    αρνητικά ορισμένος. Έτσι το γινόμενο του θετικά ορισμένου \( Q \) με τον
    αρνητικά ορισμένο \( A \) θα μας δώσει έναν αρνητικά ορισμένο πίνακα.
    Άρα ισχύει
    \[
        \frac{dV}{dt} = x^{T}QAx < 0,
    \]
    για κάθε \( x \neq 0 \). Τέτοιος \( Q \) υπάρχει πάντα καθώς η ευστάθεια του
    \( 0 \) εξαρτάται αποκλειστικά από τον πίνακα \( A \). Ακόμα το γινόμενο
    \( QA \) μπορεί να γραφτεί σαν ένας πίνακας \( P \) που θα είναι θετικά
    ορισμένος και έτσι η παραπάνω σχέση μπορεί να γραφτεί
    \[
        \frac{dV}{dt} = -x^{T}Px.
    \]
    Ένα τέτοιο παράδειγμα στο \( \mathbb{R}^3 \) είναι το παρακάτω. Ο \( Q \)
    είναι
    \[
        Q =
        \begin{pmatrix}
            2 & -1 & 0 \\
            -1 & 2 & -1 \\
            0 & -1 & 2
        \end{pmatrix},
    \]
    που έχει ιδιοτιμές \( \lambda_1 = 3.4141, \lambda_2 = 2, \lambda_3 = 0.5857
    \) και άρα είναι θετικά ορισμένος. Ο \( P \) είναι
    \[
        P =
        \begin{pmatrix}
            2 & -1 & 1 \\
            -1 & 2 & -1 \\
            1 & -1 & 2
        \end{pmatrix},
    \]
    που έχει ιδιοτιμές \( \lambda_1 = 1, \lambda_2 = 4, \lambda_3 = 1 \), που
    επίσης έχει θετικές ιδιοτιμές και άρα είναι θετικά ορισμένος. Ο πίνακας
    \( A \) είναι
    \[
        A =
        \begin{pmatrix}
            -1.25 & 0 & -0.75 \\
            -0.5 & -1 & -0.5 \\
            -0.75 & 0 & -1.25
        \end{pmatrix},
    \]
    που έχει ιδιοτιμές \( \lambda_1 = -1, \lambda_2 = -0.5, \lambda_3 = -2 \), που
    έχει αρνητικές ιδιοτιμές και άρα είναι αρνητικά ορισμένος.
\end{solution}
