\begin{exercise}{2016/17 5}
    \begin{enumerate}[label= (\alph*)]
        \item Μελετήστε το σύστημα στο επίπεδο
            \[
                \dot{x} = 2x - x^2, \quad \dot{y} = -y + xy.
            \]
            Βρείτε τα σημεία ισορροπίας, τον τύπο τους, καθώς και τις ευσταθείς
            και ασταθείς πολλαπλότητες καθενός από αυτά.

            Η \emph{θεωρία της δομικής ευστάθειας} λέει ότι, γενικά, κάθε ασταθής
            πολλαπλότητα ενός σημείου ισορροπίας (ή οριακού κύκλου) τέμνει την
            ευσταθή πολλαπλότητα άλλου σημείου ισορροπίας ή οριακού κύκλου
            \emph{εγκάρσια}.

            Δείξτε ότι στο παράδειγμα αυτό η τομή δεν είναι εγκάρσια. Είναι
            λοιπόν φυσικό να αναμένουμε ότι μικρές διαταραχές καταστρέφουν την
            μη-εγκάρσια αυτή τομή. Περιγράψτε τρόπους να γίνει αυτό.
            \emph{Υπόδειξη: επιλέξτε σημείο πάνω στην τροχιά που συνδέει τα
                σημεία ισορροπίας και με χρήση του κυτίου ροής (\tl{flow box})
            αλλάξτε τοπικά το διανυσματικό πεδίο}.
        \item Αναγνωρίστε παρόμοιες μη-εγκάρσιες τομές αναλλοίωτων πολλαπλοτήτων
            στις Χαμιλτονιανές περιπτώσεις των συστημάτων \tl{Duffing} και απλού
            εκκρεμούς.

            Συζητήστε κατά πόσο μπορούμε να φέρουμε τα συστήματα αυτά σε
            \enquote*{γενική θέση}, μένοντας όμως στην κατηγορία των
            Χαμιλτονιανών συστημάτων.
    \end{enumerate}
\end{exercise}
\begin{solution}{2016/17 5}

\end{solution}
